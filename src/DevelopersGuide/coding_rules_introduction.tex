% Copyright 2005-2016 Airbus-EDF-IMACS-Phimeca
% Permission is granted to copy, distribute and/or modify this document
% under the terms of the GNU Free Documentation License, Version 1.2
% or any later version published by the Free Software Foundation;
% with no Invariant Sections, no Front-Cover Texts, and no Back-Cover
% Texts.  A copy of the license is included in the section entitled "GNU
% Free Documentation License".

\emph{We would like to thank Vincent Lefebvre and Olivier Morvant from the Descartes project, who are the original authors of this document that we have almost entirely drawn upon.}

\subsubsection{Document goals}
The coding rules proposed in this document mainly regard:
\begin{itemize}
\item the structure of modules and classes;
\item the names of the source files;
\item the use of the C++ and Python programming languages;
\item the documentation included inside the source files;
\item the format used to write statements in the programming languages.
\end{itemize}

The purpose of these rules is to provide a common formalism for the project in order to help the developers produce code that is globally homogeneous, readable, understandable by all and also efficient and reliable, respecting the architectural choices and constraints established to produce the code of the \OT\ project. The scope of this document only embraces files produced in the \OT\ project and does not apply to external software included in the project \emph{as is}.

\subsubsection{Conventions}
This paragraph establishes the conventions used in this document.

Each coding rules is identified by a unique number prefixed by the letter R.
\RuleX{xx}{Description of rule}
Examples of source code in C++ and Python are framed. Correct examples are given in blue, while incorrect examples are in red.

\lstset{language=C++, basicstyle=\normalsize}
\begin{lstlisting}[frame=TRL]
<Example>
...
\end{lstlisting}
\lstset{language=C++, basicstyle=\color{blue}}
\begin{lstlisting}[frame=RL]
<correct example>
...
\end{lstlisting}

\lstset{language=C++, basicstyle=\color{red}}
\begin{lstlisting}[frame=RLB]
<incorrect example>
...
\end{lstlisting}

When an example requires comments, these are given above the said example and they appear in italics.

\emph{Comment about example}
\lstset{language=C++, basicstyle=\normalsize}
\begin{lstlisting}[frame=TRBL]
<Example>
\end{lstlisting}

Elements from the text, rules or source code cited as an example and appearing between the two characters {\bf \textless} and {\bf \textgreater} correspond to a description of the expected items or objects. The following example shows that {\bf \textless condition \textgreater} represents a boolean expression and the function {\bf compute} is called on a piece of data of type length, with its default unit.

\begin{lstlisting}[frame=TRBL]
if ( <condition> ) {
r = compute( <length in cm> )
}
\end{lstlisting}

By extension, the following symbols are used to describe a grammar, a call syntax, etc.

\begin{tabular}{rl}
{\bf \textless real \textgreater} & a real number \\
{\bf \textless integer \textgreater} & an integer \\
{\bf \textless string \textgreater} & a string \\
{\bf \textless filename \textgreater} & a string representing a filename
\end{tabular}

The first rule is given below this introductory paragraph and deals with the language used in the code.
\Rule{Coding language}{{\bf Coding language:} the language used to program classes, functions and modules in \OT\ is English. The comments included in the source code shall be written in English as well.}
