% Copyright 2005-2016 Airbus-EDF-IMACS-Phimeca
% Permission is granted to copy, distribute and/or modify this document
% under the terms of the GNU Free Documentation License, Version 1.2
% or any later version published by the Free Software Foundation;
% with no Invariant Sections, no Front-Cover Texts, and no Back-Cover
% Texts.  A copy of the license is included in the section entitled "GNU
% Free Documentation License".

Coding rules for the C++ language in \OT.

\subsubsection{C++ language standard}
\Rule{Cpp-ANSI}{The use of a programming language for which there is an ISO/ANSI \index{Standard!C++}standard involves compliance with this standard. The source files are compiled, whenever possible, with the options carrying out the ISO/ANSI standard compliance check.}

\emph{Example: GCC compilation:}
\lstset{language=C++, basicstyle=\normalsize}
\begin{lstlisting}[frame=TRBL]
g++ -c -ansi ...
\end{lstlisting}
\emph{Note: Compliance with the ISO/ANSI standard sometimes introduces significant limitations on the use of basic definitions or basic system functions. Moreover, strict checks on the compliance with the ANSI standard are switched on through options that are compiler-specific.}

\subsubsection{Namespaces}
\Rule{Cpp-namespace}{The \OT\ C++ types, classes and functions will be declared in an {\bf OpenTURNS} \index{Namespace}namespace, which will be aliased to {\bf OT}.}

\emph{Example of {\bf OpenTURNS} namespace definition for simple types:}
\lstset{language=C++, basicstyle=\normalsize}
\begin{lstlisting}[frame=TBRL]
/ **
* @file       OTtypes.hxx
* ...
* /
namespace OT
{
/* < Declarations of simple types > * /

/* < Declarations of objects and functions > * /
} ;

// Alias for the direct use of XXX
namespace OpenTURNS = OT;
\end{lstlisting}

\emph{Example of a class declaration in the {\bf OpenTURNS} namespace and in a nested namespace}

\emph{{\bf Uncertainty::Model}:}
\lstset{language=C++, basicstyle=\normalsize}
\begin{lstlisting}[frame=TBRL]
/**
* @file      RandomVector.hxx
* ...
*/
namespace OT {
class RandomVector {
/* ... */
}; /* end class RandomVector */
} /* namespace OT */
\end{lstlisting}

\emph{Example of use by making all the definitions contained in the namespace available:}
\lstset{language=C++, basicstyle=\normalsize}
\begin{lstlisting}[frame=TBRL]
#include "OT.hxx"
using namespace OT ;

void f( NumericalScalar n ) ;
\end{lstlisting}

\Rule{Cpp-no-using}{Employing the {\bf using} directive is prohibited in header files.}

\emph{Example of namespace use:}
\lstset{language=C++, basicstyle=\normalsize}
\begin{lstlisting}[frame=TBRL]
#include "OT.hxx"

void f( UnsignedLong n );
\end{lstlisting}

\subsubsection{Names}
\Rule{Cpp-names}{\index{Type!simple}\index{Simple type}Simple (or \index{Type!built-in}\index{Built-in type}built-in) types are recoded, using the {\bf typedef} instruction, in the {\bf <OTtypes.hxx>} file included in all C++ source files.}

\emph{Example:}
\lstset{language=C++, basicstyle=\normalsize}
\begin{lstlisting}[frame=TBRL]
#ifndef OPENTURNS_OTTYPES_HXX
#define OPENTURNS_OTTYPES_HXX

#include <string>
/**
* basic types mapping
*/
namespace OT
{
typedef  bool          Bool ;

typedef  unsigned long UnsignedLong ;

typedef  double        NumericalScalar ;

typedef  std::string   String ;

/* ... */
} ;

#endif /* OPENTURNS_OTTYPES_HXX */
\end{lstlisting}

\Rule{Cpp-capitalization}{When a \index{Type!name}type name (simple, structure, class) is defined by the programmer, each word composing the name must begin with an uppercase letter.}

\emph{Example:}
\lstset{language=C++, basicstyle=\normalsize}
\begin{lstlisting}[frame=TBRL]
class NumericalSample {
...
} ; /* end class NumericalSample */
\end{lstlisting}

\Rule{Cpp-variables}{The names of \index{Variable!name}variables outside of classes must begin with a lowercase letter.}

\emph{Example:}
\lstset{language=C++, basicstyle=\normalsize}
\begin{lstlisting}[frame=TBRL]
int main( ) {
Bool myCondition ;
...
}
\end{lstlisting}

\Rule{Cpp-class-members}{The names of all class members must end with an underscore.}

\emph{Example:}
\lstset{language=C++, basicstyle=\normalsize}
\begin{lstlisting}[frame=TBRL]
class Environment : public Object {
...
private :
NumericalScalar density_ ; //<! material density in environment (g/cm3)
...
} ; /* end class Environment */
\end{lstlisting}
NB: It is common for the underscore to be used as a prefix for private attribute names. However, this method may trigger conflicts with internal variables or definitions used by the compilers. For this reason, the underscore is used as a suffix.

\Rule{Cpp-capitalize-member}{The first character of the names of \index{Member data!of class}class \index{Class!member data}member data ({\bf static} type \index{Attribute!name}\index{Static!attribute name}attributes) must be an uppercase letter. For any other attribute, it will be a lowercase letter.}

\emph{Example:}
\lstset{language=C++, basicstyle=\normalsize}
\begin{lstlisting}[frame=TBRL]
class Object {
...
private :
static  String ClassName_ ;
...
} ; /* end class Object */
\end{lstlisting}

\Rule{Cpp-capitalize-static-methods}{The name of a {\bf static} \index{Method!name}\index{Static!method name}class method or a {\bf static} function must begin with an uppercase letter. The names of other methods and functions begin with a lowercase letter.}

\emph{Example:}
\lstset{language=C++, basicstyle=\normalsize}
\begin{lstlisting}[frame=TBRL]
class Object
{
public:
//* returns a class identifier based on its name
static  String GetClassName() ; ...
} ; /* end class Object */
\end{lstlisting}

\Rule{Cpp-capitalize-static-members}{The name of a {\bf static} \index{Variable!name}\index{Static!variable name}variable must begin with an uppercase letter.}

\emph{Example:}
\lstset{language=C++, basicstyle=\normalsize}
\begin{lstlisting}[frame=TBRL]
int
initializeConversion( )
{
static Bool IsInitialized = false ;
if ( ! IsInitialized ) {
...
IsInitialized = true ;
}
} ;
\end{lstlisting}

\Rule{Cpp-constants}{The name of a \index{Constant!name}constant must consist of a series of \emph{uppercase} words separated by underscores.}

\emph{Example:}
\lstset{language=C++, basicstyle=\normalsize}
\begin{lstlisting}[frame=TBRL]
const UnsignedLong MAXIMUM_OF_RETRIES = 5;
\end{lstlisting}

\Rule{Cpp-name-words}{The \index{Type!name}names of types and \index{Variable!name}variables must consist of a sequence of \emph{whole} words. From the second word on, they begin with an uppercase letter.}

\emph{Example:}
\lstset{language=C++, basicstyle=\normalsize}
\begin{lstlisting}[frame=TBRL]
int main()
{
NumericalScalar reactionRate = 0. ;
...
}
\end{lstlisting}

\Rule{Cpp-verb}{The names of \index{Function!name}functions and \index{Method!name}methods must begin with an action verb in the infinitive form. This verb begins with a lowercase letter. From the second word on, the words begin with an uppercase letter.}

\emph{Example:}
\lstset{language=C++, basicstyle=\normalsize}
\begin{lstlisting}[frame=TBRL]
class NumericalSample
{
UnsignedLong getDimension ( ) const ;
...
} ; /* end class NumericalSample */
\end{lstlisting}

\Rule{Cpp-no-abbrev}{The names of variables, members, functions, methods and classes must not contain any \index{Abbreviation}abbreviation, except when they are self-explanatory and can not be confusing.}

\emph{Example:}
\lstset{language=C++, basicstyle=\color{blue}}
\begin{lstlisting}[frame=TBRL]
class NumericalSample {
} ; /* end class NumericalSample */

void removeElement(const UnsignedLong index) ;

NumericalPoint meanValue ;
\end{lstlisting}

\emph{Example of tolerated notations:}
\lstset{language=C++, basicstyle=\color{blue}}
\begin{lstlisting}[frame=TBRL]
UnsignedLong i ;                // loop indices i, j and k are common
UnsignedLong j ;
UnsignedLong k ;

UnsignedLong nbMaxElements ;    // usual abbreviations: nb, Max

void
addPoint( NumericalPoint pt ) { // no confusion in the context
add( pt ) ;
}
\end{lstlisting}

\emph{Incorrect examples:}
\lstset{language=C++, basicstyle=\color{red}}
\begin{lstlisting}[frame=TBRL]
NumericalScalar a, k, l, m1, m2, m3 ;
NumericalScalar zzz, zz2;
const char *foo, *hello, tempo, bogus ;

void adElt( NumericalPoint pt );

UnsignedLong numSmplPt ;
\end{lstlisting}

\subsubsection{Class declaration}
\Rule{Cpp-class-visibility}{The constituents of a class should be declared in the following order: {\bf \index{Public}public}, {\bf \index{Protected}protected}, {\bf \index{Private}private}. All static methods and members are placed before anything else.}

\emph{Example:}
\lstset{language=C++, basicstyle=\normalsize}
\begin{lstlisting}[frame=TBRL]
class Buffer {
public :
static AThing GetThing();
protected:
private :
static AThing TheThing_;

public :
NumericalScalar getValue() const;
protected :
NumericalScalar theValue_;
private :
/* ... */
} ; /* end class Buffer */
\end{lstlisting}

\Rule{Cpp-class-structure}{The \index{Class declaration!basic structure}basic structure for a class declaration is as follows: declaration of a default \index{Constructor}constructor, of a \index{Copy constructor}copy constructor, of a virtual \index{Destructor}destructor, of a \index{Copy assignment}copy assignment operator and if necessary of a \index{Comparison}comparison operator ({\bf '=='}) and of a stream converter, followed by the other methods of the class.}

\emph{Example:}
\lstset{language=C++, basicstyle=\normalsize}
\begin{lstlisting}[frame=TBRL]
class AnyClass {
public :
/** Default constructor  */
AnyClass( ) ;
/** Copy constructor */
AnyClass( const AnyClass & other ) ;
/** Destructor */
virtual ~AnyClass( ) ;
/** Copy operator  */
AnyClass& operator = ( const AnyClass & other ) ;
/** Comparison operator */
Bool operator == ( const AnyClass & other ) const ;
/** Stream converter */
String repr( ) const ;
String str( ) const ;

/* other public methods ... */

private :
UnsignedLong size_ ;
DataType * data_ ;

/* other private methods ... */
} ; /* end class AnyClass */
\end{lstlisting}


\Rule{Cpp-private-attr}{All \index{Attribute!private}attributes of a class should be placed in the {\bf \index{Protected}protected} or {\bf \index{Private}private} sections. For classes designed to be derived, attributes can be promoted to the {\bf \index{protected}protected} \index{Attribute!protected}section.}

\emph{Example:}
\lstset{language=C++, basicstyle=\normalsize}
\begin{lstlisting}[frame=TBRL]
class AnyClass {
public :
/* ... */
private :
UnsignedLong size_ ;
DataType * data_ ;
} ; /* end class AnyClass */
\end{lstlisting}

\Rule{Cpp-attr-initialization}{All \index{Attribute!initialization}attributes must be initialized in the constructors, using their respective constructors and respecting the order of attribute declaration in the class.}

\emph{Example:}
\lstset{language=C++, basicstyle=\normalsize}
\begin{lstlisting}[frame=TBRL]
class Vector {
public :
Vector ( Bool someProperty, UnsignedLong size, NumericalScalar elt = 0. ) ;
private :
Bool property_;
Collection<NumericalScalar> data_ ;
} ;
\end{lstlisting}

\emph{Example of a correct definition:}
\lstset{language=C++, basicstyle=\color{blue}}
\begin{lstlisting}[frame=TBRL]
Vector::Vector ( Bool someProperty, UnsignedLong size, NumericalScalar elt )
: property_(someProperty), data_( size, elt )
{ }
\end{lstlisting}

\emph{Examples of incorrect definitions:}
\lstset{language=C++, basicstyle=\color{red}}
\begin{lstlisting}[frame=TBRL]
Vector::Vector ( Bool someProperty, UnsignedLong size, NumericalScalar elt )
: data_(size, elt), property_(someProperty)     // order of initialization
{ }

Vector::Vector ( Bool someProperty, UnsignedLong size, NumericalScalar elt )
{
property_ = someProperty;
data_ = Collection<NumericalScalar>(size, elt);
// requires an assignment after the construction
// processing is longer for complex objects!
}
\end{lstlisting}

\Rule{Cpp-no-virtual-in-ctor}{The constructors of a class must not call {\bf \index{Virtual}virtual} type methods of the class.}

\emph{Example: declaration of a pure virtual class A and of class B, derived from A:}
\lstset{language=C++, basicstyle=\color{blue}}
\begin{lstlisting}[frame=TBRL]
class A {
public :
A( ) ;                  // constructor
virtual ~A( );          // destructor
virtual const char * getClassName( ) = 0;       // pure virtual method
} ;

class B : public A {
public :
const char * getClassName( ) { return "B" ; }
} ;
\end{lstlisting}

\emph{Incorrect definitions leading to an execution error:}
\lstset{language=C++, basicstyle=\color{red}}
\begin{lstlisting}[frame=TBRL]
A::A( ) {
cout << getClassName( ) << " created" << endl ; // B does not exist yet!
}

A::~A( ) {
cout << getClassName( ) << " destroyed" << endl ; // B no longer exists!
}

B::B( ) : A( )
{ }
\end{lstlisting}

\Rule{Cpp-accessors}{
If an attribute is readable and/or writable by the outside world and/or derived classes, the \index{Access method}access methods given below are reported under the {\bf public} or {\bf protected} sections.
For access methods:
\begin{itemize}
\item \index{Type!simple}\index{Simple type}Simple types or "built-in types" ({\bf bool}, {\bf int}, {\bf float}, {\bf long}, {\bf double}, ...) are passed or returned as values;
\item Composite \index{Type!composite}\index{Composite type}types (class, structure) are passed as constant references. The return type depends on the implementation of the class. No reference to temporaries nor non-const reference to members should be returned.
\end{itemize}
}

\emph{Write method for the {\bf name} attribute:}
\lstset{language=C++, basicstyle=\normalsize}
\begin{lstlisting}[frame=TBRL]
void            setName ( SimpleType ) ;
void setName    ( const ComposedType & ) ;
\end{lstlisting}
\emph{Read method for the {\bf name} attribute:}
\lstset{language=C++, basicstyle=\normalsize}
\begin{lstlisting}[frame=TBRL]
SimpleType              getName( ) const ;
const ComposedType &    getName( ) const ;
\end{lstlisting}
\emph{Example:}
\lstset{language=C++, basicstyle=\normalsize}
\begin{lstlisting}[frame=TBRL]
class NumericalSample {
public :
//* return the dimension of the sample
UnsignedLong getDimension( ) const ;

//* return the i-th element
NumericalPoint         operator[] (UnsignedLong i);
const NumericalPoint & operator[] (UnsignedLong i) const;
} ;
\end{lstlisting}

\Rule{Cpp-inline}{
The bodies of {\bf \indexfunction{inline}inline} type methods may be located outside of the class declaration, in the header file, in order to make the class easier to read.
}

\emph{Example:}
\lstset{language=C++, basicstyle=\normalsize}
\begin{lstlisting}[frame=TBRL]
class NumericalSample {
public :
//* return the number of the rod
inline UnsignedLong getDimension( ) const { return dimension_; }

//* compute the mean point of the sample
inline NumericalPoint computeMeanValue() const;
};

//* inline methods
NumericalPoint computeMeanValue() const;
{
/* ... some non trivial processing */
return meanValue;
}
\end{lstlisting}


\subsubsection{Inheritance}
\Rule{Mutliple-inheritance}{
\index{Multiple inheritance}\index{Inheritance!multiple}Multiple inheritance should be used \emph{only when necessary} and its use \emph{must be justified}.
}

\Rule{Call-ctor-in-derived}{
The constructor of the parent class must be called in the constructor of the derived class prior to any member initialization.
}

\emph{Example: the Point class derives from the Vector class}
\lstset{language=C++, basicstyle=\normalsize}
\begin{lstlisting}[frame=TBRL]
class NumericalPoint : public std::vector<double> {
public:
Point ( NumericalScalar x, NumericalScalar y,
NumericalScalar z );
} ;

Point::Point( NumericalScalar x, NumericalScalar y,
NumericalScalar z )
: std::vector<double>( 3 )
{
(*this)[0] = x ;
(*this)[1] = y ;
(*this)[2] = z ;
}
\end{lstlisting}

\Rule{Virtual-dtor}{
Except for classes designed not to be derived, the \index{Destructor!virtual}\index{Virtual!destructor}destructor of a class must systematically be declared as {\bf virtual}, in order to ensure that the destructors of potential derived classes are actually called.
}

\emph{Example:}
\lstset{language=C++, basicstyle=\normalsize}
\begin{lstlisting}[frame=TBRL]
class Object {
public :
Object( );
virtual ~Object( );
} ;
\end{lstlisting}

\subsubsection{Function and method declaration}
\Rule{Declared-functions}{
Each function that is not local to a compilation unit must be subjected to a prototype declaration in an included file.
}

\Rule{Doxygen-declarations}{
The prototype of a function or method must be as follows:
}

\lstset{language=C++, basicstyle=\normalsize}
\begin{lstlisting}[frame=TBRL]
/** @brief <short description>
*
* <Long description>
* @param argument_1 <description>
* @param argument_2 <description>
* @return           <description>
* @throw            <description>
*/
ReturnType
functionName (
TypeArgument_1       argument_1,
TypeArgument_2   argument_2
);
\end{lstlisting}


\Rule{Object-by-ref}{
For structured (i.e. different from simple types) types, \index{Parameter passing by reference}parameter passing should occur by \emph{reference} rather than by value. Structured types that are not modified by a method or a function must be declared as {\bf const}.
}
\emph{Correct example:}
\lstset{language=C++, basicstyle=\color{blue}}
\begin{lstlisting}[frame=TBRL]
void send( const String & message ) ;
\end{lstlisting}
\emph{Incorrect Example:}
\lstset{language=C++, basicstyle=\color{red}}
\begin{lstlisting}[frame=TBRL]
void send( String message ) ;
\end{lstlisting}

\Rule{Const-methods}{
Methods that do not modify the class instance should be declared as {\bf \indexfunction{const}const}.
}

\Rule{Function-overloading}{
For a function or a method, overloading must be used preferably to a variable number of arguments or optional arguments.
}
\emph{Correct example:}
\lstset{language=C++, basicstyle=\color{blue}}
\begin{lstlisting}[frame=TBRL]
Buffer & append( UnsignedLong );
Buffer & append( const String & );
Buffer & append( NumericalScalar ) ;
\end{lstlisting}
\emph{Incorrect Example:}
\lstset{language=C++, basicstyle=\color{red}}
\begin{lstlisting}[frame=TBRL]
Buffer & append( const char *fmt, ... ) ;
Buffer & append( const char *str = 0, double d = 0., int i = 0 ) ;
\end{lstlisting}

\Rule{Template-usage}{
\emph{Inline} or \emph{template} \indexfunction{inline}\index{Template}functions should be used rather than macro-functions.
}

\subsubsection{Variable declaration}

\Rule{One-declaration-per-line}{
Do not declare more than one \index{Variable!declaration}variable per line. \emph{Always remember to initialize variables and to add a comment for each}.
The declaration of \emph{multiple} variables on one line is allowed if \emph{all variables are of the same type}.
}
\emph{Correct example:}
\lstset{language=C++, basicstyle=\color{blue}}
\begin{lstlisting}[frame=TBRL]
String         filename ("") ; // library file name
UnsignedLong   nbElements(0) ; // number of elements into the data file
UnsignedLong   i = 0;
UnsignedLong   j = 0;
\end{lstlisting}
\emph{Accepted example:}
\lstset{language=C++, basicstyle=\color{blue}}
\begin{lstlisting}[frame=TBRL]
UnsignedLong   i = 0, j = 0, k = 0;     // indices
\end{lstlisting}
\emph{Incorrect Example:}
\lstset{language=C++, basicstyle=\color{red}}
\begin{lstlisting}[frame=TBRL]
/ * Multiple declarations and different types * /
UnsignedLong   i, j, tab[20], **l, *numberOfElements ;
String        filename, *yourname, myname ;
\end{lstlisting}

\subsubsection{Constant declaration}
\Rule{Const-var-definition}{
The use of {\bf const} variables must be preferred over {\bf \#define}.
}
\emph{Example:}
\lstset{language=C++, basicstyle=\color{blue}}
\begin{lstlisting}[frame=TBRL]
const UnsignedLong maximumIterations = 32;
const char printFormat[] = "%s:line %d, %s" ;
\end{lstlisting}
\emph{Incorrect Example:}
\lstset{language=C++, basicstyle=\color{red}}
\begin{lstlisting}[frame=TBRL]
#define MAXIMUM_ITERATIONS 32;
#define PRINT_FORMAT       "%s:line %d, %s"
\end{lstlisting}

\subsubsection{Comments and internal documentation }
\Rule{Beginning-comment}{
An introductory comment must appear at the beginning of each file. In the case of a class declaration file, this comment will follow the structure below:
}
\lstset{language=C++, basicstyle=\normalsize}
\begin{lstlisting}[frame=TBRL]
/**
* @file        file name and version
* @brief       short description
*
* <LGPL copyright>
*
* @project    OpenTURNS
*
* @author first name LAST NAME (login)
* @date   Wed Nov 24 15:32:07 MET 1997
*
* Copyright (C) EDF-EADS-Phimeca 2005-YYYY
*/
\end{lstlisting}

\Rule{Comment-function-arg}{
There must be a comment for each parameter of a function or method. See rule R\ref{rule:Doxygen-declarations}.
}

\Rule{Comment-when-necessary}{
Explanatory comments precede, if necessary, a group of instructions. These comments should explain "why" these instructions are there, not "how" they were implemented, except if a very specific optimization or implementation requires further explanation.
}
These comments should avoid:
\begin{itemize}
\item mentioning the names of variables;
\item being a simple transcription of the code into English.
\end{itemize}


\subsubsection {Memory allocation and deallocation}
This section discusses general rules for \index{Memory!allocation}allocating and \index{Memory!freeing}freeing memory. It will later be supplemented by rules regarding the use of basic classes in order to manage the lifecycle of objects in memory.

\Rule{Use-auto-var}{
The definition of \index{Variable!automatic}automatic variables must be favored over dynamic allocation. It is preferable to use the STL containers ({\bf list}, {\bf vector}, {\bf map}, {\bf queue}, etc.).
}
\emph{Example to favor:}
\lstset{language=C++, basicstyle=\color{blue}}
\begin{lstlisting}[frame=TBRL]
{
NumericalScalar sections1[MAX] ; // a fixed size array
vector<NumericalScalar> sections2 ; // an extensible vector
list<Volume> volumes ; // a list of volumes

/* ... */
}
\end{lstlisting}
\emph{Example to avoid:}
\lstset{language=C++, basicstyle=\color{red}}
\begin{lstlisting}[frame=TBRL]
{
NumericalScalar *sections = new NumericalScalar[MAX] ;
list<Volume>    *volumes  = new list<Volume> ;

/* ... */

delete [ ] sections ;
delete volumes ;
}
\end{lstlisting}

\Rule{Prefer-new-delete-over-malloc-free}{
In the C++ programs, the operators \indexfunction{new}{\bf new} and \indexfunction{delete}{\bf delete} must be used instead of the {\bf malloc} and {\bf free} functions when initializing memory.
}
\emph{Correct example:}
\lstset{language=C++, basicstyle=\color{blue}}
\begin{lstlisting}[frame=TBRL]
{
Volume *volume = new Volume ;   // memory allocation +
// constructor call
/* ... */
delete volume ;                 // destructor call +
volume = 0 ;                    // memory deallocation
}
\end{lstlisting}
\emph{Incorrect example:}
\lstset{language=C++, basicstyle=\color{red}}
\begin{lstlisting}[frame=TBRL]
{
Volume *volume = (Volume*)malloc(sizeof(Volume)) ;
// memory allocation but
// no constructor call
/* ... */
free( volume ) ;                // no destructor call before
volume = 0 ;                    // memory deallocation
}
\end{lstlisting}

\Rule{Delete-array}{
In order to free the entire memory space allocated to an object vector, the string {\bf '[]'} must not be forgotten after the {\bf delete} operator, in order to call the destructor on all of the objects placed in the table. For restrictions to this rule, see rule R\ref{rule:Smart-pointer}.
}
\emph{Example:}
\lstset{language=C++, basicstyle=\normalsize}
\begin{lstlisting}[frame=TRL]
A *     a = new A[40] ; // calls the constructor 40 times
...
\end{lstlisting}
\lstset{language=C++, basicstyle=\color{red}}
\begin{lstlisting}[frame=RL]
delete a ;              // incorrect: the table is freed,
// the ~A destructor isn't called
\end{lstlisting}

\lstset{language=C++, basicstyle=\color{blue}}
\begin{lstlisting}[frame=RLB]
delete [] a ;           // correct: the table is freed,
// the ~A destructor is called 40 times
\end{lstlisting}

\Rule{Smart-pointer}
{To avoid having to manually manage the \index{Memory!management}memory allocation and deallocation, the use of \index{Smart pointer}"smart pointers" is mandatory. It is \emph{prohibited} to directly use C pointers. The OpenTURNS {\bf Pointer<T>} class implements a smart pointer for class {\bf T}. For restrictions to this rule, see rule R\ref{rule:C-pointer-in-interface}.
}
\emph{List of declaration files for the smart pointer:}
\lstset{language=C++, basicstyle=\normalsize}
\begin{lstlisting}[frame=TBRL]
#include "Pointer.hxx"
\end{lstlisting}

\Rule{C-pointer-in-interface}{
Though it is forbidden to directly use C pointers (which should thus never appear in the interface of an object), there are a few tolerated exceptions. Among these, methods to convert objects to C structures are allowed. However, specific attention must be paid to avoiding memory leaks.
}


\subsubsection{Assignment and initialization}

\Rule{Prefer-initialization}{
Initializations should be preferred over assignments.
}

\emph{Example:}
\lstset{language=C++, basicstyle=\color{blue}}
\begin{lstlisting}[frame=TBRL]
String message( "finished" ) ;

String message = "done" ;
\end{lstlisting}

\emph{Example to avoid:}
\lstset{language=C++, basicstyle=\color{red}}
\begin{lstlisting}[frame=TBRL]
String message ;
message = "Computation complete" ; // assignment after construction

String message() ; // declaration of a function prorotype
\end{lstlisting}

\Rule{Unordered-initialization}{
Do not make assumptions about the execution order of a sequence of initializations between different compilation units.
}

\subsubsection{Instructions}
\Rule{Emacs-indentation}{
The \index{Indentation}indentation follows Emacs' indentation mode.
}

\Rule{Code-for-clarity}{
Whenever efficiency constraints allow it, the production of \emph{clear code} must be preferred over optimized code. When it is not possible, comment on the optimization and the resulting constraints.
}

\Rule{Not-too-deep-nests}{
Limit to a reasonable number (\emph{maximum 3}) the number of \index{Nested control structures}\index{Control structures, nested}nested control structures. In the case of complex algorithms, one must use specific functions, possibly templates, to limit the depth of nesting.
}

\Rule{One-statement-per-line}{
For readability, one must strive to write only one statement per line. {\bf For} loops are not affected by this rule.
}
\emph{Example:}
\lstset{language=C++, basicstyle=\normalsize}
\begin{lstlisting}[frame=TBRL]
i = 0 ;
while ( i < MAX ) {
++i ;
f(i) ;
}
\end{lstlisting}

\emph{Examples to avoid if possible:}
\lstset{language=C++, basicstyle=\color{red}}
\begin{lstlisting}[frame=TBRL]
a = b = c = 0 ;
// multiple assignments

f(++i) ;
// readability

v = *i++ ;
// performance and understandability

for( i = 1, j = 2, k = 3; i < N; j++, i++ );
// understandability and readability
\end{lstlisting}

\emph{Incorrect examples:}
\lstset{language=C++, basicstyle=\color{red}}
\begin{lstlisting}[frame=TBRL]
buffer += "test", cout << buffer ; i = 1 ;
// heterogeneous processing &
// different objects

while( f(++i), i < MAX);
// processing carried out before the test
\end{lstlisting}

\Rule{No-goto}{
\emph{The use of the \indexfunction{goto}{\bf goto} instruction must be prohibited}, even for error handling.
}

\emph{Prohibited example:}
\lstset{language=C++, basicstyle=\color{red}}
\begin{lstlisting}[frame=TBRL]
void foo( ) {
for( ... ) {
phase1 :
/* ... */
phase2 :
if( errno != 0 )
goto erreur ;
if ( /* a test */ )
goto phase2 ;
}
erreur :
/* error handling */
}
\end{lstlisting}

\emph{Note: error handling can be easily replaced with an exception handling, and the use of {\bf goto} for the needs of algorithms can always be replaced with a conditional structure or a loop.}

\Rule{Minimize-temporaries}{
Minimize the number of \index{Temporary objects}temporary objects in a block of instructions. Declarations should occur at the latest moment in order to avoid useless object creation.
}
\emph{Example:}
\lstset{language=C++, basicstyle=\color{blue}}
\begin{lstlisting}[frame=TBRL]
NumericalScalar
compute( UnsignedLong n ) {
NumericalScalar result ;
if( n < MIN || n > MAX ) {
char msg[BUFSIZ];
// automatic allocation for the processing
sprintf ( msg,
"n = %d is out of range, valid range is [%d, %d]",
n, MIN, MAX );
throw Exception( msg ) ;
}
/* ... */
return result;
}
\end{lstlisting}

\emph{Examples to avoid:}
\lstset{language=C++, basicstyle=\color{red}}
\begin{lstlisting}[frame=TBRL]
NumericalScalar
compute( UnsignedLong n ) {
NumericalScalar result ;
Char    msg[BUFSIF] ;   // allocation unnecessary if no
// error
if( n < MIN || n > MAX )
...
}
\end{lstlisting}


\Rule{Avoid-switch}{
The {\bf switch} keyword should be avoided. Prefer design, virtualization and polymorphism. When {\bf switch} is used, the \index{switch!default}{\bf default} case must always be present. In a {\bf switch}, the instructions for the different {\bf case} items must always end with a {\bf break}. The cases will always use constants defined in advance.
}
\emph{Correct example:}
\lstset{language=C++, basicstyle=\color{blue}}
\begin{lstlisting}[frame=TRL]
switch ( errno ) {
case ENOENT :
msg = " ... ";
\end{lstlisting}
\lstset{language=C++, basicstyle=\color{blue}}
\begin{lstlisting}[frame=BRL]
break ;
case EACCESS :
msg = " ... " ;
break ;
default :
msg = "unknown error" ;
break ;
}
\end{lstlisting}

\emph{Accepted example - processing multiple targets with the same block:}
\lstset{language=C++, basicstyle=\color{blue}}
\begin{lstlisting}[frame=TBRL]
switch ( errno ) {
case ENOENT :
case EACCESS :
msg = " impossible to access file " ;
break ;
default :
/* ... */
break ;
}
\end{lstlisting}

\emph{Incorrect example:}
\lstset{language=C++, basicstyle=\color{red}}
\begin{lstlisting}[frame=TBRL]
switch ( errno ) {
case 1 :                // it is a value
msg = " ... " ; //
// "break" is missing,
// processing continues in ENOENT
case ENOENT :
msg = " ... " ;
break ;
default :               // "break" is missing at the
// end of the "default" case
msg = "unknown error" ;
}
\end{lstlisting}

\emph{Incorrect example - use of the switch as a goto:}
\lstset{language=C++, basicstyle=\color{red}}
\begin{lstlisting}[frame=TBRL]
switch ( phase ) {
case PHASE1 :
doPhaseOne() ;
case PHASE2 :
doPhaseTwo() ;
break ;
default :
/* ... */
}
\end{lstlisting}

\Rule{Main-return}{
Programs must always end with a return code via the \index{exit}{\bf exit (<integer>)} function or via the {\bf return <integer>;} instruction in the main function, using the {\bf EXIT\_SUCCESS} or {\bf EXIT\_FAILURE} values. The {\bf EXIT\_FAILURE} constant may replaced, if necessary, with a catalogued \index{Return code}return code whose value is between 2 and 127. Numbers starting from 128 are reserved for the operating system.
}
\emph{Example:}
\lstset{language=C++, basicstyle=\normalsize}
\begin{lstlisting}[frame=TBRL]
int
main ( int argc, char *argv[] )
{
/* ... */
return EXIT_SUCCESS ;
}
\end{lstlisting}

\subsubsection{Exceptions}

The ability to raise and handle \index{Exception}exceptions is one of the strongest features of C++. However, writing functions and methods that guarantee a safe behavior when faced with exceptions remains a difficult aspect of programming.

This chapter describes how to define and use exceptions in the source code.

\Rule{Exception-safety}{
When writing a function, the developer will bear in mind that the code must first perform all operations that may throw an exception, and then perform operations that change the state of an object or of the system (see \emph{Exception-Safety Issues and Techniques, pages 25-68, Exceptional C++}).
}

\Rule{Exception-hierarchy}{
Exceptions used by the \OT\ code are inherited from the general class {\bf Exception}. Other types of exceptions are simply processed.
}
\emph{Example of {\bf Exception} use}
\lstset{language=C++, basicstyle=\normalsize}
\begin{lstlisting}[frame=TBRL]
class Exception {
public :
Exception ( const char *description, const char * comment = 0 ) ;
virtual ~Exception( ) throw();
/* ... */
friend ostream & operator<< (ostream &, const Exception & e ) ;
} ;
\end{lstlisting}

\emph{Example of specialization of {\bf Exception} in order to report an off-range error}
\lstset{language=C++, basicstyle=\normalsize}
\begin{lstlisting}[frame=TBRL]
class OutOfBoundException : public Exception {
public:
OutOfBoundException( /* ... */ )
: Exception( /* ... */ ) { }
} ;
\end{lstlisting}

\emph{Example of specialization of {\bf Exception} in order to report an off-range error with a macro-instruction}
\lstset{language=C++, basicstyle=\normalsize}
\begin{lstlisting}[frame=TBRL]
NEW_EXCEPTION( OutOfBoundException );
\end{lstlisting}

\Rule{No-throw-specifier}{
C++ functions and methods throwing an exception should not specify it in the signature of the method or function using the \index{Exception!throw}{\bf throw} keyword. It is better to mention it in the Doxygen comment describing the function or method (see rule R\ref{rule:Doxygen-declarations}). Furthermore, it is not recommended to pass an exception from function to function. If the error can not be processed, the exception must be transformed into a new, more explicit one, whose meaning is adapted to the function's level of knowledge.
}

\Rule{Exception-for-error}{
Exceptions are reserved for error processing. Using an exception for a jump or as direct output of a function must be prohibited.
}
\emph{Incorrect Example:}
\lstset{language=C++, basicstyle=\color{red}}
\begin{lstlisting}[frame=TRL]
try {
// phase 1
// phase 2
if ( result != OK )
throw GotoPhase4() ;
// phase 3
\end{lstlisting}
\lstset{language=C++, basicstyle=\color{red}}
\begin{lstlisting}[frame=BRL]
/* ... */
catch ( GotoPhase4 e ) {
/* ... */
}
// phase 4
\end{lstlisting}

\Rule{Dtor-no-throw}{
The \index{Destructor}destructor of an object must not throw any exception.
}

\Rule{RAII}{
The programming structure of \emph{resource acquisition is initialization} \index{Resource acquisition is initialization}(RAII) must be used preferably in order to get rid of complex processing when raising an exception.
}

\Rule{Catch-exceptions-near-throw}{
Processing an exception should be made as close as possible to the function or method that throws this exception.
}

\subsubsection{Error handling and error messages}

\Rule{}{
Information, warning and error messages are constructed using a message number and the class corresponding to the message level (respectively {\bf Log::User}, {\bf Log::Warning} and {\bf Log::Error}). Messages can be enhanced with comments during construction or by using the redirection operator.
}
\emph{Example:}
\lstset{language=C++, basicstyle=\normalsize}
\begin{lstlisting}[frame=TBRL]
String message;
Log::Debug( message );
Log::Wrapper( message );
Log::Info( message );
Log::User( message );
Log::Warning( message );
Log::Error( message );
\end{lstlisting}
