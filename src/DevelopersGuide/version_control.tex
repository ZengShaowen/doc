\label{version_control}

This section describes how to use the OpenTURNS version control system AKA subversion repository. These instructions may be useful for OpenTURNS developers which have an account on subversion.



\subsection{File hierarchy}

\subsubsection{Description}

As usual with subversion repository, root directory contains three directories described bellow :
\begin{itemize}
\item trunk
\item tags
\item branches
\end{itemize}

Each user have a personal branch named simply by username. Every branch must be follow some rules to be more easy to merge with others developments. These rules are really simple:


\begin{itemize}
\item At the first time, the developer must create one devel directory by copying the trunk using a conventional commit message (see below);
\item if a developer wants to make some support on one or more major version, he must copy the corresponding tags, this time using a special commit message;
\item IT IS STRICTLY FORBIDDEN for a developer to apply a patch manually across subversion directories: he must ALWAYS use the subverion merging facility to do that;
\item IT IS STRICTLY FORBIDDEN for a developer to merge a branch with another one: all the merges must be done from or to the trunk.
\end{itemize}



For example, with one developer "John Doh", the file hierarchy looks like this:

\begin{itemize}
\item trunk
\item tags
\begin{itemize}
\item openturns-0.12.1
\item openturns-0.12.2
\item openturns-0.12.3
\item openturns-0.13.0
\end{itemize}
\item branches
\begin{itemize}
\item doh
\end{itemize}
\end{itemize}

\subsubsection{An example}

It is simpler to explain how each directory is used through an example. Suppose that a new developer, John Doh, joins the OpenTURNS development team:
upcoming developer in OpenTURNS repository :
\begin{itemize}
\item On the server side :

\begin{enumerate}
\item Subversion's administrator creates a new account named "doh" for the user John Doh;
\item Subversion's administrator creates a new subdirectory /doh in /branches, leading to /branches/doh: it is the 'private' user branch;
\end{enumerate}

Now, John Doh can start its work in the repository. The first action for the developer is to "branch" the trunk, which is the reference for the project: the trunk is the place for the latest working development version.

\item On the client side :

\begin{enumerate}

\item John Doh can now "checkout" its private branch into a local copy, i.e. on its local computer, not on the server.

\begin{lstlisting}
svn checkout https://svn.openturns.org/openturns/branches/doh doh-branch
\end{lstlisting}

\item To be able to develop with the latest version, John copies the trunk's content into his local copy.

You may have multiple contiguous directories in your branch to split your developments according to their targets.
Here we suppose that the developments will be merged in the trunk, for example to add a new functionality.

\begin{lstlisting}
svn copy https://svn.openturns.org/trunk doh-branch
\end{lstlisting}

\item The previous step has made a local copy of the trunk into the local copy of its branch. Now, John MUST make his first commit :

\begin{lstlisting}
svn commit -m "BRANCH: svn copy https://.../trunk@1170 branch"
\end{lstlisting}

To ease the managment of the version control system, we enforce the systematic use of comments for each commit. When the commit is related to a branch modification such as a copy, a merge, a delete, a tag or anything similar, the comment MUST start by the corresponding keyword in uppercase. These keywords are:

\begin{itemize}
\item BRANCH for a branch creation or deletion
\item MERGE for a merge between a branche and the trunk (in either direction)
\item TAG for a tag
\end{itemize}
Also, when you create a branch you have to specify the associated trunk revision number (here, r1170). You can find it at http://trac.openturns.org/browser
It is the number in the "rev" column.


\item Each step (ie functionality, bug fix, etc.) must be commited separately with a comment. e.g. if John Doh write an enhanced cache system he writes:

\begin{lstlisting}
svn commit -m "Introduced an enhanced cache system."
\end{lstlisting}

The -m option is useful for single line comments, ie for very small comments. If your commit needs to be explained in a few lines, it is better to log the entire comments in a plain text file (e.g., svn.log is a quite common name) and to use the -F option followed with the file name.
\begin{lstlisting}
edit svn.log
svn commit -F svn.log
\end{lstlisting}

\end{enumerate}
\end{itemize}


\subsection{Merging process}

This section describes the several steps of the merging process, in particular the preparation work that must be done on the developer side before the integrator can merge its developments.

The whole process must succeed in order to be able to merge. If any failure occurs, it MUST be fixed, otherwise the subversion repository would be corrupted. Suppose that John Doh wants to merge its branch with the latest revision of the trunk, nammely the revision 620 (r620). He must:
\begin{enumerate}
\item Checkout the branch to merge (ie: for John Doh)

\begin{lstlisting}
svn co https://svn.openturns.org/branches/doh/devel doh-devel
\end{lstlisting}

\item Find the latest synchronization revision of this branch with trunk (e.g. r365). he finds this information by looking at the latest MERGE or BRANCH keyword in the log of his branch
\begin{lstlisting}
svn log | egrep -B6 -e 'BRANCH|MERGE'
\end{lstlisting}

Here we suppose that the latest synchronization was done at the revision 365 (r365).
\item Apply the differences between the latest synchronization (r365) and the latest revision of trunk (r620) to the local copy of the branch

\begin{lstlisting}
cd doh-devel
svn merge -r365:620 https://svn.openturns.org/openturns/trunk
\end{lstlisting}

\item Verify patched files and conflicts

\begin{lstlisting}
svn status | egrep -e '^C'
svn status | egrep -v -e '^[DGURAM]'
\end{lstlisting}

These commands should not return any error or missing files, i.e. the output should be empty.
\item Verify that the compilation and the archive creation process works fine with this updated version of the branch

\begin{lstlisting}
mkdir build
cd build
cmake .. -DCMAKE_INSTALL_PREFIX=$PWD/install
make
make check
make install
make installcheck
cd -
\end{lstlisting}

In this process, it is required to build and test the developments in a directory SEPARATED from the source directory. So DO make a build subdirectory and DO change do it. It does not take time but it ensures that the building process is correct (you can't check for this when building in the same directory than the sources).
\item If there is no error, commit this revision with a special log message

\begin{lstlisting}
svn commit -m 'MERGE: svn merge -r365:620 https://.../trunk'
\end{lstlisting}

When commiting, don't forget to add the correct keyword (MERGE, BRANCH, TAG): it will be crucial for the next merge. Otherwise it is very painful to find the latest synchronization point.

\end{enumerate}

\subsection{Tagging and releasing process }

This section explains how the integrator must tag and release a new version with this version control system.

Add this step, we suppose that the merging process has been done. The trunk is clean and is ready to be released as an official version.

\subsubsection{Tagging}
\begin{enumerate}
\item The first step, before tagging, is to update the ChangeLog file:
\begin{lstlisting}
svn log https://svn.openturns.org/openturns/trunk -r rev_start:rev_end|grep -v "^r"|grep -v "^--"|grep -v "^[ \t]*$$" >> log
\end{lstlisting}

\item Set the new release version
\begin{lstlisting}
./utils/setVersionNumber.sh 1.X
\end{lstlisting}

\item You can now commit these modifications
\begin{lstlisting}
svn ci -m 'Next release (1.X) preparation.'
\end{lstlisting}

\item Finally, you can now tag the new version with the following command
\begin{lstlisting}
svn cp https://svn.openturns.org/openturns/trunk https://svn.openturns.org/openturns/tags/openturns-1.X -m 'TAG: This is official release 1.X.'
\end{lstlisting}

\end{enumerate}
\subsubsection{Releasing}

\begin{enumerate}

\item Export the newly created tag in a temporary directory
\begin{lstlisting}
cd /tmp
svn export https://svn.openturns.org/tags/openturns-1.X
\end{lstlisting}

\item Build the tarball

\begin{lstlisting}
cd openturns-1.X
mkdir build
cd build
cmake ..
make package_source
\end{lstlisting}

\end{enumerate}
