% Copyright 2005-2016 Airbus-EDF-IMACS-Phimeca
% Permission is granted to copy, distribute and/or modify this document
% under the terms of the GNU Free Documentation License, Version 1.2
% or any later version published by the Free Software Foundation;
% with no Invariant Sections, no Front-Cover Texts, and no Back-Cover
% Texts.  A copy of the license is included in the section entitled "GNU
% Free Documentation License".
\renewcommand{\filename}{docUC_LoadSave_Save.tex}
\renewcommand{\filetitle}{UC : How to save a study ?}

% \HeaderNNIILevel
\HeaderIILevel
% \HeaderIIILevel

The objective of this Use Case is to explicitate how to save the structures created within a study in order to be able to load them in a future study.\\


\index{Study!Save}
\requirements{
  \begin{description}
  \item[$\bullet$] none
  \end{description}
}
             {
               \begin{description}
               \item[$\bullet$] an object containing all the objects saved : {\itshape myStudy}
               \item[type:] Study
               \item[$\bullet$]  a file .XML containing all the objects saved : {\itshape myXMLFile.XML}
               \item[type:] file .XML
               \end{description}
             }

             \textspace\\
             Python  script for this UseCase :

             \begin{lstlisting}
               # Create the name of the file .XML which will be created at the adress /tmp
               # if the adress is not precised, the file .XML is created in the current repertory
               fileName = "/tmp/myXMLFile"

               # Create a Study Object which will contain all the objects saved
               myStudy = Study()

               # Associate it to the file .XML just created
               myStudy.setStorageManager(XMLStorageManager(fileName))

               # Perform here the study
               # for example, a NumericalPoint is created we want to save
               p1 = NumericalPointWithDescription( ( ("x",10.), ("y",11.), ("z",12.) ) )

               # Add the NumericalPoint to the list of the objects saved
               myStudy.add("point", p1)

               # Create an analytical NumericalMathFunction
               analytical = NumericalMathFunction( ("a", "b"), ("sum", "prod", "mean"), ("a+b", "a*b", "(a+b)/2") )

               # Add the NumericalMathFunction to the list of the objects saved
               myStudy.add("analytical", analytical)

               # Check the list of objects that will be saved
               print "Study = " , myStudy

               # Remove the NumericalMathFunction to the list of the objects saved
               myStudy.remove("analytical")

               # CARE!! At this point, no object has been saved : only the list have been written!
               # SAVE NOW the objects listed
               myStudy.save()
             \end{lstlisting}
