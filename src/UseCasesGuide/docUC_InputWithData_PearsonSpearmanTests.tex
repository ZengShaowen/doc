% Copyright (C) 2005-2015 Airbus - EDF - IMACS - Phimeca
% Permission is granted to copy, distribute and/or modify this document
% under the terms of the GNU Free Documentation License, Version 1.2
% or any later version published by the Free Software Foundation;
% with no Invariant Sections, no Front-Cover Texts, and no Back-Cover
% Texts.  A copy of the license is included in the section entitled "GNU
% Free Documentation License".
\renewcommand{\filename}{docUC_InputWithData_PearsonSpearmanTests.tex}
\renewcommand{\filetitle}{UC : Particular manipulations of the Pearson and Spearman tests, when the first sample is of dimension superior to 1.}

% \HeaderNNIILevel
% \HeaderIILevel
\HeaderIIILevel


\index{Independence Test!ChiSquared test}
\index{Independence Test!Pearson test}
\index{Independence Test!Spearman test}

The objective of this Use Case is to decide whether two samples follow a monotonous or linear relation in the case where the first sample is of dimension $>1$.\\
The Pearson and Spearman tests are evaluated successively between some (or all) coordinates of the first sample and the second one, which must be of dimension 1.\\


Details on the Pearson and Spearman tests  may be found in the Reference Guide (\extref{ReferenceGuide}{see files Reference Guide - Step B -- Pearson correlation test, Step B -- Spearman correlation test}{stepB}).\\


\requirements{
  \begin{description}
  \item[$\bullet$] one continuous scalar numerical sample of dimension n : {\itshape sample1}
  \item[type:]  NumericalSample
  \item[$\bullet$] one continuous scalar numerical sample of dimension 1 : {\itshape sample2}
  \item[type:]  NumericalSample
  \end{description}
}
             {
               \begin{description}
               \item[$\bullet$] tests results : {\itshape resultPartialPearson, resultFullPearson, resultPartialSpearman, resultFullSpearman}
               \item[type:] TestResultCollection
               \end{description}
             }

             \textspace\\
             Python script for this UseCase :

             \inputscript{script_docUC_InputWithData_PearsonSpearmanTests}
