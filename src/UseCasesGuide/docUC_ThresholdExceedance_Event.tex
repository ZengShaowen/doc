% Copyright 2005-2016 Airbus-EDF-IMACS-Phimeca
% Permission is granted to copy, distribute and/or modify this document
% under the terms of the GNU Free Documentation License, Version 1.2
% or any later version published by the Free Software Foundation;
% with no Invariant Sections, no Front-Cover Texts, and no Back-Cover
% Texts.  A copy of the license is included in the section entitled "GNU
% Free Documentation License".
\renewcommand{\filename}{docUC_ThresholdExceedance_Event.tex}
\renewcommand{\filetitle}{UC : Creation of an event in the physical and the standard spaces}

% \HeaderNNIILevel
% \HeaderIILevel
\HeaderIIILevel

\label{StandardPhysicalEvent}

\index{Event!Physical space}
\index{Event!Standard space}


This section gives elements to create events in the physical space {\itshape Event} and in the standard space {\itshape StandardEvent}.\\



Details on isoproabilitic transformations  may be found in the Reference Guide (\extref{ReferenceGuide}{see files Reference Guide - Step C -- Isoprobabilistic transformation preliminary to FORM-SORM methods}{stepC}).\\


The below script creates the event based on the scalar ouput variable  {\itshape output} and defined by :
\begin{align*}
  myEvent = \{ output > 4\}.
\end{align*}

\requirements{
  \begin{description}
  \item[$\bullet$] the scalar  output variable of interest : {\itshape output}
  \item[type:] RandomVector which implementation is a CompositeRandomVector
  \end{description}
}
             {
               \begin{description}
               \item[$\bullet$] the events in the physical and standard spaces : {\itshape myEvent}, {\itshape myStandardEvent}
               \item[type:] Event and StandardEvent
               \end{description}
             }

             \textspace\\
             Python script for this UseCase :

             \inputscript{script_docUC_ThresholdExceedance_Event}
