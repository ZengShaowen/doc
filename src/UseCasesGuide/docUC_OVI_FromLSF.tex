% Copyright (C) 2005-2015 Airbus - EDF - IMACS - Phimeca
% Permission is granted to copy, distribute and/or modify this document
% under the terms of the GNU Free Documentation License, Version 1.2
% or any later version published by the Free Software Foundation;
% with no Invariant Sections, no Front-Cover Texts, and no Back-Cover
% Texts.  A copy of the license is included in the section entitled "GNU
% Free Documentation License".
\renewcommand{\filename}{docUC_OVI_FromLSF.tex}
\renewcommand{\filetitle}{UC : Creation of the ouput random vector}

% \HeaderNNIILevel
% \HeaderIILevel
\HeaderIIILevel

\label{CompositeRandomVector}


\index{Random Vector!Output random vector}

The objective of this Use Case is to create the ouput random variable of interest defined as the image through the limit state function of the input random vector.\\

Details on the definition of random mixture variables may be found in the Reference Guide (\extref{ReferenceGuide}{see files Reference Guide - Step B -- Random Mixture : affine combination of independent univariate distributions}{stepB}).\\

\requirements{
  \begin{description}
  \item[$\bullet$] the limit state function : {\itshape myFunction}
  \item[type:]  NumericalMathFunction
  \item[$\bullet$] the random input vector : {\itshape inputRV}
  \item[type:] RandomVector which implementation is a UsualRandomVector
  \end{description}
}
             {
               \begin{description}
               \item[$\bullet$] the output variable of interest {\itshape outputRV = myFunction(inputRV)}
               \item[type:] RandomVector which implementation is a CompositeRandomVector
               \end{description}
             }

             \textspace\\
             Python script for this UseCase :

             \inputscript{script_docUC_OVI_FromLSF}
