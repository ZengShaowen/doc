% Copyright (C) 2005-2015 Airbus - EDF - IMACS - Phimeca
% Permission is granted to copy, distribute and/or modify this document
% under the terms of the GNU Free Documentation License, Version 1.2
% or any later version published by the Free Software Foundation;
% with no Invariant Sections, no Front-Cover Texts, and no Back-Cover
% Texts.  A copy of the license is included in the section entitled "GNU
% Free Documentation License".
\renewcommand{\filename}{docUC_InputWithData_RegressionTest.tex}
\renewcommand{\filetitle}{UC : Regression test between two scalar numerical samples}

% \HeaderNNIILevel
% \HeaderIILevel
\HeaderIIILevel



\index{Regression Linear Model!Rsquared@$R^2$ test}



The objective of this Use Case is to detect a linear relation between two scalar numerical samples. \\



Details on the linear regression model  may be found in the Reference Guide (\extref{ReferenceGuide}{see file Reference Guide - Step B -- Linear regression}{stepB}).\\



\requirements{
  \begin{description}
  \item[$\bullet$] one continuous scalar numerical sample of dimension n : {\itshape Xsample}
  \item[type:]  NumericalSample
  \item[$\bullet$] one continuous scalar numerical sample of dimension 1 : {\itshape Ysample}
  \item[type:]  NumericalSample
  \end{description}
}
             {
               \begin{description}
               \item[$\bullet$] tests results : {\itshape resultPartialRegression, resultFullRegression, resultPartialSpearman, resultFullSpearman}
               \item[type:] TestResultCollection
               \end{description}
             }

             \textspace\\
             Python script for this UseCase :

             \inputscript{script_docUC_InputWithData_RegressionTest}
