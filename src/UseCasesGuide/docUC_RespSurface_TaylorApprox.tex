% Copyright 2005-2016 Airbus-EDF-IMACS-Phimeca
% Permission is granted to copy, distribute and/or modify this document
% under the terms of the GNU Free Documentation License, Version 1.2
% or any later version published by the Free Software Foundation;
% with no Invariant Sections, no Front-Cover Texts, and no Back-Cover
% Texts.  A copy of the license is included in the section entitled "GNU
% Free Documentation License".
\renewcommand{\filename}{docUC_RespSurface_TaylorApprox.tex}
\renewcommand{\filetitle}{UC : Linear and Quadratic Taylor approximations}

% \HeaderNNIILevel
% \HeaderIILevel
\HeaderIIILevel

\label{taylorApprox}

\index{Response Surface!Linear Taylor approximation}
\index{Response Surface!Quadratic Taylor approximation}
\index{Response Surface!Linear least squares approximation}

This section details the first method to construct a response surface : from the linear or quadratic Taylor approximations of the function at a particular point.\\

Details on response surface approximations may be found in the Reference Guide (\extref{ReferenceGuide}{see files Reference Guide - Step Res. Surf. -- Polynomial Response Surfaces : principles and -- Taylor Expansion}{responseSurface}).\\


\requirements{
  \begin{description}
  \item[$\bullet$] a function : {\itshape myFunc}
  \item[type:]  NumericalMathFunction
  \end{description}
}
             {
               \begin{description}
               \item[$\bullet$] the linear Taylor approximation {\itshape myLinearTaylor}
               \item[type:]  LinearTaylor
               \item[$\bullet$] the quadratic Taylor approximation {\itshape myQuadraticTaylor}
               \item[type:]  QuadraticTaylor
               \end{description}
             }

             \textspace\\
             Python  script for this UseCase :

             \begin{lstlisting}
               # Taylor approximations at point 'center'
               center = NumericalPoint(myFunc.getInputNumericalPointDimension())
               for i in range(center.getDimension()) :
               center[i] = 1.0+i

               # Create the linear Taylor approximation
               myLinearTaylor = LinearTaylor(center, myFunc)

               # Create the quadratic Taylor approximation
               myQuadraticTaylor = QuadraticTaylor(center, myFunc)

               # Perform the approximations
               # linear Taylor approximation
               myLinearTaylor.run()
               print "my linear Taylor =" , myLinearTaylor

               # quadratic Taylor approximation
               myQuadraticTaylor.run()
               print "my quadratic Taylor =" , myQuadraticTaylor

               # Stream out the result
               # linear Taylor approximation
               linearResponseSurface = myLinearTaylor.getResponseSurface()
               print "responseSurface =" , linearResponseSurface

               # quadratic Taylor approximation
               quadraticResponseSurface = myQuadraticTaylor.getResponseSurface()
               print "quadraticResponseSurface =" , quadraticResponseSurface

               # Compare the approximations and the function at a particluar point
               # point 'center'
               print "myFunc(" , center , ")=" , myFunc(center)
               print "linearResponseSurface(" , center , ")=" , linearResponseSurface(center)
               print "quadraticResponseSurface(" , center , ")=" , quadraticResponseSurface(center)
             \end{lstlisting}
