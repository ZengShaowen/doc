% Copyright (C) 2005-2015 Airbus - EDF - IMACS - Phimeca
% Permission is granted to copy, distribute and/or modify this document
% under the terms of the GNU Free Documentation License, Version 1.2
% or any later version published by the Free Software Foundation;
% with no Invariant Sections, no Front-Cover Texts, and no Back-Cover
% Texts.  A copy of the license is included in the section entitled "GNU
% Free Documentation License".
\renewcommand{\filename}{docUC_LSF_wrapper.tex}
\renewcommand{\filetitle}{UC : From an external wrapper with gradient and hessian implementations}

% \HeaderNNIILevel
% \HeaderIILevel
\HeaderIIILevel



\index{Limit State Function!External wrapper}

The objective of this Use Case is to specify the limit state function, defined through an external wrapper .\\

Details on how to build a wrapper may be found in the Developers Guide (\extref{DevelopersGuide}{see file Wrapper Guide}{wrapperDev}).\\

The example here is the wrapper {\itshape poutre.xml} which contains the implementations of :
\begin{itemize}
\item the function $func\_exec\_compute\_deviation$,
\item its gradient $grad\_exec\_compute\_deviation$ and
\item its hessian $hes\_exec\_compute\_deviation$.
\end{itemize}
\vspace*{0.5cm}

It is necessary to refer to the documentation {\itshape OpenTURNS - Wrappers Guide} to have explanations on what constitues an OpenTURNS wrapper. \\
It is possible to separate the loading of the wrapper file and the creation of the NumericalMathFunction, as indicated in CASE 2 of the script given belows.

\requirements{
  \begin{description}
  \item[$\bullet$] wrapper of the limit state function {\itshape poutre.xml}
  \end{description}
}
             {
               \begin{description}
               \item[$\bullet$] the limit state function : {\itshape poutre}(*)
               \item[type:]  NumericalMathFunction
               \end{description}
             }

             (*) :
             \begin{equation}
               \label{equatPoutre}
               \begin{array}{l|lcl}
                 poutre : & \Rset^4 & \rightarrow & \Rset \\
                 & (E,F,L,I)    & \mapsto     & y_0 = \displaystyle \frac{FL^3}{3EI}
               \end{array}
             \end{equation}

             \textspace\\
             Python script for this UseCase :

             \begin{lstlisting}
               # CASE 1 : we load the wrapper file and create the NumericalMathFunction at the same time
               # Create the limit state function 'poutre' from the wrapper 'poutre'
               poutre = NumericalMathFunction("poutre")

               # CASE 2 : we load separately the wrapper file and create the NumericalMathFunction
               # Load the wrapper file
               wrap = WrapperFile.FindWrapperByName("poutre")
               # Create the limit state function 'poutre' from the wrapper wrap
               poutre = NumericalMathFunction(wrap)
             \end{lstlisting}
