% Copyright (C) 2005-2015 Airbus - EDF - IMACS - Phimeca
% Permission is granted to copy, distribute and/or modify this document
% under the terms of the GNU Free Documentation License, Version 1.2
% or any later version published by the Free Software Foundation;
% with no Invariant Sections, no Front-Cover Texts, and no Back-Cover
% Texts.  A copy of the license is included in the section entitled "GNU
% Free Documentation License".
\renewcommand{\filename}{docUC_RespSurface_Kriging.tex}
\renewcommand{\filetitle}{UC : Kriging metamodelling approximation from a design experiment}

% \HeaderNNIILevel
% \HeaderIILevel
\HeaderIIILevel

\label{krigingApprox}

\index{Response Surface!Kriging}

This Use Case details the method to build a response surface from a design experiment by Kriging.




\requirements{
  \begin{description}
  \item[$\bullet$] a sample of the input vector: {\itshape X}
  \item[type:] NumericalSample
  \item[$\bullet$] a sample of the output vector: {\itshape Y}
  \item[type:] NumericalSample
  \end{description}
}
             {
               \begin{description}
               \item[$\bullet$] the polynomial chaos algorithm: {\itshape algo}
               \item[type:] a KrigingAlgorithm
               \item[$\bullet$] the meta-model function: {\itshape metamodel}
               \item[type:] a NumericalMathFunction
               \end{description}
             }

             \textspace\\
             Python script for this Use Case :

             \inputscript{script_docUC_RespSurface_Kriging}
