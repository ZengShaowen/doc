% Copyright (C) 2005-2015 Airbus - EDF - IMACS - Phimeca
% Permission is granted to copy, distribute and/or modify this document
% under the terms of the GNU Free Documentation License, Version 1.2
% or any later version published by the Free Software Foundation;
% with no Invariant Sections, no Front-Cover Texts, and no Back-Cover
% Texts.  A copy of the license is included in the section entitled "GNU
% Free Documentation License".
\renewcommand{\filename}{docUC_EventProcess.tex}
\renewcommand{\filetitle}{UC : Creation of an  event based on a process}

% \HeaderNNIILevel
% \HeaderIILevel
\HeaderIIILevel

\label{EventProcess}

\index{Stochastic Process!Event based on a process}


This section gives elements to create  an event based on a multivariate stochastic process.\\

Let $X: \Omega \times \cD \rightarrow \Rset^d$ be a stochastic process of dimension $d$, where $\cD \in \Rset^n$ is discretized on the mesh $\cM$. We suppose that $\cM$ contains $N$ vertices.\\

We define the event $\cE$ as:
\begin{align}\label{eventProcStoch}
  \displaystyle \cE(X) = \bigcup_{\vect{t}\in \cM}\left\{X_{\vect{t}}  \in \cA  \right\}
  % \cE(X) = \{ \exists \vect{t} \in \cD \, | \, X_{\vect{t}}  \in \cA \}
\end{align}
where $\cA$ is a domain of $\Rset^d$. \\

A particular domain $\cA$ is the cartesian product of type :
\begin{align*}
  \cA = \prod_{i=1}^d [a_i,b_i]
\end{align*}
In that case, OpenTURNS defines $\cA$ by its both extreme points : $\vect{a}$ and $\vect{b}$.\\

In the general case, $\cA$ is a {\itshape Domain} object that is able to check if it contains a given point in $\Rset^d$.\\

OpenTURNS creates an {\itshape Event} object from the process $X$ and the domain $\cA$. Then, it is possible to get a realization of the event $\cE$, which is scalar $1_{\cE(X)}(\vect{x}_0, \dots, \vect{x}_{N-1})$ if $(\vect{x}_0, \dots,\vect{x}_{N-1})$ is a realization of $X$ on $\cM$. \\

\requirements{

  \begin{description}
  \item[$\bullet$] the  domain $\cA$: {\itshape myDomainA}
  \item[type:] Domain
  \end{description}

  \begin{description}
  \item[$\bullet$] the process : {\itshape myProcess}
  \item[type:] Process
  \end{description}
}
{
  \begin{description}
  \item[$\bullet$] the Event $\cE$: {\itshape myEvent}
  \item[type:] Event
  \end{description}

}

\textspace\\
Python script for this UseCase :

\inputscript{script_docUC_StocProc_Event}

\textspace\\
