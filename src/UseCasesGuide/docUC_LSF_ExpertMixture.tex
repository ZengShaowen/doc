% Copyright (C) 2005-2015 Airbus - EDF - IMACS - Phimeca
% Permission is granted to copy, distribute and/or modify this document
% under the terms of the GNU Free Documentation License, Version 1.2
% or any later version published by the Free Software Foundation;
% with no Invariant Sections, no Front-Cover Texts, and no Back-Cover
% Texts.  A copy of the license is included in the section entitled "GNU
% Free Documentation License".
\renewcommand{\filename}{docUC_LSF_ExpertMixture.tex}
\renewcommand{\filetitle}{UC : Defining a piece wise function according to a classifier}

% \HeaderNNIILevel
% \HeaderIILevel
\HeaderIIILevel


\label{ExpertMixture}

\index{Limit State Function!ExpertMixture}


The objective of this Use Case is define a piece wise function according to a classifier:
\begin{eqnarray}\label{expMixtFct}
  f(\vect{x})  & = f_1(\vect{x}) \quad \forall \vect{x} \in Classe\, 1 \\ \nonumber
  & = f_k(\vect{x}) \quad \forall \vect{x} \in Classe\, k  \\\nonumber
  & = f_N(\vect{x}) \quad \forall \vect{x} \in Classe\, N
\end{eqnarray}
where the $N$ classes are defined by the  classifier.



The classifier is MixtureClassifier based on a  MixtureDistribution defined as:
\begin{equation}\label{MixtDist}
  p(\vect{x}) = \sum_{i=1}^N w_ip_i(\vect{x})
\end{equation}

The rule to assign a point to a class is defined as follows:  $\vect{x}$ is assigned to the class $i=\argmax_k \log w_kp_k(\vect{x})$.\\
The grade of $\vect{x}$ with respect to the classe $k$ is $\log w_kp_k(\vect{x})$.\\

The example here is a bivariate classifier that classes points among 2 classes, using the mixture distribution defined by:
\begin{equation}\label{MixtDist}
  p(\vect{x}) = \frac{1}{2}(\phi_1(\vect{x}) + \phi_2(\vect{x}))
\end{equation}
with $\phi_i$ the probability density function of the $Normal(\vect{\mu}, \vect{\sigma}, \mat{R}_i)$ where $\vect{\mu}=(-1, 1)^t$, $\vect{\sigma}=(1, 1)^t$, $\mat{R}_1[0,1]=-0.99$ and $\mat{R}_2[0,1]=0.99$.
The  function $f$ is defined by:
\begin{eqnarray}\label{expMixtFctEx}
  f(\vect{x})  & = -\vect{x} \quad \forall \vect{x} \in Classe\, 1 \\ \nonumber
  & = +\vect{x} \quad \forall \vect{x} \in Classe\, 2
\end{eqnarray}

\requirements{-
}
             {
               \begin{description}
               \item[$\bullet$] the {\itshape moe}: model of expert mixture
               \item[type:] ExpertMixture
               \end{description}
             }

             \textspace\\
             Python script for this Use Case :

             \inputscript{script_docUC_LSF_ExpertMixture}
