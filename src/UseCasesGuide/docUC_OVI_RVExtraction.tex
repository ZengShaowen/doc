% Copyright (C) 2005-2015 Airbus - EDF - IMACS - Phimeca
% Permission is granted to copy, distribute and/or modify this document
% under the terms of the GNU Free Documentation License, Version 1.2
% or any later version published by the Free Software Foundation;
% with no Invariant Sections, no Front-Cover Texts, and no Back-Cover
% Texts.  A copy of the license is included in the section entitled "GNU
% Free Documentation License".
\renewcommand{\filename}{docUC_OVI_RVExtraction.tex}
\renewcommand{\filetitle}{UC : Extraction of a random subvector from a random vector}

% \HeaderNNIILevel
% \HeaderIILevel
\HeaderIIILevel


\index{Random Vector!Extracting a sub vector}

The objective of this Use Case is to extract a subvector from a random vector which has been defined as well  as a UsualRandomvector (it means thanks to a distribution, see UC. \ref{UsualRandomVector}) than  as a CompositeRandomVector (as the image through a limit state function of an input  random vector, see UC. \ref{CompositeRandomVector}).\\

Let's note $\vect{Y} = (Y_1, \cdots, Y_n)$ a random vector and $I \subset [1, n]$ a set of indices :
\begin{itemize}
\item In the first case, the subvector is defined by $\vect{\tilde{Y}} = (Y_i)_{i \in I}$,
\item In the second case, where $\vect{Y} = f(\vect{X})$ with $f = (f_1, \cdots, f_n)$, $f_i$ some scalar functions, the sub vector is $\vect{\tilde{Y}} = (f_i(\vect{X}))_{i \in I}$.
\end{itemize}


\noindent%
\requirements{
  \begin{description}
  \item[$\bullet$] the random vector : {\itshape myRandomVector}
  \item[type:] RandomVector which implementation is a UsualRandomVector or CompositeRandomVector
  \end{description}
}
             {
               \begin{description}
               \item[$\bullet$] the extracted random vector : {\itshape myExtractedRandomVector}
               \item[type:] RandomVector which implementation is a UsualRandomVector or CompositeRandomVector
               \end{description}
             }

             \textspace\\
             Python script for this UseCase :


             \begin{lstlisting}

               # CASE 1 : Get the marginal of the random vector
               # Corresponding to the component i

               # Care : numerotation begins at 0
               myExtractedRandomVector = myRandomVector.getMarginal(i)


               # CASE 2 : Get the marginals of the random vector
               # Corresponding to several components
               # decribed in the myIndice table
               # For example, components 0, 1, and 5

               myExtractedRandomVector = myRandomVector.getMarginal( (0, 1, 5) )
             \end{lstlisting}
