% Copyright (C) 2005-2015 Airbus - EDF - IMACS - Phimeca
% Permission is granted to copy, distribute and/or modify this document
% under the terms of the GNU Free Documentation License, Version 1.2
% or any later version published by the Free Software Foundation;
% with no Invariant Sections, no Front-Cover Texts, and no Back-Cover
% Texts.  A copy of the license is included in the section entitled "GNU
% Free Documentation License".
\renewcommand{\filename}{docUC_LSF_SomeParticularFunctions.tex}
\renewcommand{\filetitle}{UC : Some particular functions : linear combination, agregation, composition}

% \HeaderNNIILevel
% \HeaderIILevel
\HeaderIIILevel


\label{NumericalMathFunction}






\index{Limit State Function!Composition}
\index{Limit State Function!Linear combination}
\index{Limit State Function!Agregated funtion}
\index{Limit State Function!Gradient}
\index{Limit State Function!Hessian}

The objective of this Use Case is to create some particular functions :
\begin{itemize}
\item the scalar linear combination \textit{linComb} of vectorial functions $vectFctColl = (f_1, \hdots, f_N)$ where
  \begin{align*}
    \forall 1 \leq i \leq N, \,     f_i : \Rset^n \longrightarrow \Rset^{p}
  \end{align*}
  with specific scalar weights $scalWeight = (c_1, \hdots, c_N)\in \Rset^N $ :
  \begin{align*}
    \begin{array}{l|lcl}
      linComb : & \Rset^n & \longrightarrow & \Rset^{p} \\
      &  \vect{X} & \longrightarrow & \displaystyle \sum_{i=1}^N c_if_i (\vect{X})
    \end{array}
  \end{align*}


\item the vectorial linear combination \textit{vectLinComb} of a set of  functions $scalFctColl = (f_1, \hdots, f_N)$ where
  \begin{align*}
    \forall 1 \leq i \leq N, \,     f_i : \Rset^n \longrightarrow \Rset
  \end{align*}
  with specific vectorial weights $vectWeight = (\vect {c}_1, \hdots, \vect{c}_N)$  where
  \begin{align*}
    \forall 1 \leq i \leq N, \,   \vect{c}_i \in \Rset^p
  \end{align*}
  \begin{align*}
    \begin{array}{l|lcl}
      vectLinComb : & \Rset^n & \longrightarrow & \Rset^{p} \\
      &  \vect{X} & \longrightarrow & \displaystyle \sum_{i=1}^N \vect{c}_if_i (\vect{X})
    \end{array}
  \end{align*}

\item the agregated function $h$ of a set of functions $(f_1, \hdots, f_N)$ where
  \begin{align*}
    f_i : \Rset^n \longrightarrow \Rset^{p_i}
  \end{align*}
  \begin{align*}
    \begin{array}{l|lcl}
      agregFct : & \Rset^n & \longrightarrow & \Rset^{p} \\
      &  \vect{X} & \longrightarrow & (f_1(\vect{X}), \hdots, f_N(\vect{X}))^t
    \end{array}
  \end{align*}
  with
  \begin{align*}
    p = \displaystyle \sum_{i=1}^N p_i
  \end{align*}

\item the indicator function $l$ of an event defined by a function $f$, a comparison operator and a threshold $s$. For example, if the comparison operator is $>$, then
  \begin{align*}
    l = 1_{\{f > s\}}
  \end{align*}
\end{itemize}


OpenTURNS automatically evaluates the analytical expressions of the gradient and the hessian, except if the analytical expression of the function is not differentiable everywhere. In that case, OpenTURNS implements a finite difference method :
\begin{itemize}
\item the gradient evaluation method is the centered finite difference method, with the differential increment $h=1e-5$ for each direction,
\item the hessian evaluation method is the centered finite difference method, with the differential increment $h=1e-4$ for each direction.
\end{itemize}

\requirements{
  \begin{description}
  \item[$\bullet$] some collections of scalar and vectorial functions : {\itshape scalFctColl, vectFctColl}
  \item[type:] some NumericalMathFunctionCollection
  \item[$\bullet$] a list of  scalar weights : {\itshape scalWeight}
  \item[type:] a NumericalPoint
  \item[$\bullet$] a list of  vectorial weights : {\itshape vectWeight}
  \item[type:] a NumericalSample
  \item[$\bullet$] a function : {\itshape function}
  \item[type:] a NumericalMathFunction
  \end{description}
}
             {
               \begin{description}
               \item[$\bullet$] some particular funtions : {\itshape linComb, vectLinComb, agregFct, indFactor}
               \item[type:] some NumericalMathFunction
               \end{description}
             }

             \textspace\\
             Python script for this UseCase :

             \inputscript{script_docUC_LSF_SomeParticularFunctions}
