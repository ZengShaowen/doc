% Copyright 2005-2016 Airbus-EDF-IMACS-Phimeca
% Permission is granted to copy, distribute and/or modify this document
% under the terms of the GNU Free Documentation License, Version 1.2
% or any later version published by the Free Software Foundation;
% with no Invariant Sections, no Front-Cover Texts, and no Back-Cover
% Texts.  A copy of the license is included in the section entitled "GNU
% Free Documentation License".
\renewcommand{\filename}{docUC_CentralUncertainty_MomentsEvaluation.tex}
\renewcommand{\filetitle}{UC : Moments evaluation of a random sample of the output variable of interest}

% \HeaderNNIILevel
% \HeaderIILevel
\HeaderIIILevel




\index{Sample Statistics!Moments evaluation}
\index{Graph!Taylor variance decomposition importance factors}
\index{Graph Manipulation!View}
\index{Graph Manipulation!Show}

The objective of this Use Case  is to evaluate the mean and standard deviation of the output variable of interest by generating a random sample of the output variable of interest and evaluate the empirical indicators from that sample.\\

Details on empirical moments evaluation  may be found in the Reference Guide (\extref{ReferenceGuide}{see files Reference Guide - Step C -- Estimating the mean and variance using the Monte Carlo Method}{stepC}).\\

\requirements{
  \begin{description}
  \item[$\bullet$] the output variable of interest : {\itshape output}, of dimension $\geq 1$
  \item[type:] RandomVector which implementation is a CompositeRandomVector
  \end{description}
}
             {
               \begin{description}
               \item[$\bullet$] Mean and covariance of the variable of interest
               \item[type:] NumericalPoint, CovarianceMatrix
               \end{description}

               \begin{description}
               \item[$\bullet$] Covariance matrix and its Cholesky factor of the variable of interest
               \item[type:] CovarianceMatrix, SquareMatrix
               \end{description}
             }

             \textspace\\
             Python script for this UseCase :

             \inputscript{script_docUC_CentralUncertainty_MomentsEvaluation}
