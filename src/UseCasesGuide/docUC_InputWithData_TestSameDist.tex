% Copyright (C) 2005-2015 Airbus - EDF - IMACS - Phimeca
% Permission is granted to copy, distribute and/or modify this document
% under the terms of the GNU Free Documentation License, Version 1.2
% or any later version published by the Free Software Foundation;
% with no Invariant Sections, no Front-Cover Texts, and no Back-Cover
% Texts.  A copy of the license is included in the section entitled "GNU
% Free Documentation License".
\renewcommand{\filename}{docUC_InputWithData_TestSameDist.tex}
\renewcommand{\filetitle}{UC : Do two samples have the same distribution : QQ-plot visual test, Smirnov numerical test}

% \HeaderNNIILevel
% \HeaderIILevel
\HeaderIIILevel



\index{Graph!QQ-plot}
\index{Fitting Test!QQ-plot}
\index{Comparison of distribution test!Smirnov}

The objective of this Use Case is to decide whether both samples follow the same distribution or not. \\

To help the decision, OpenTURNS  proposes one visual test and one numerical test :
\begin{itemize}
\item the QQ-plot visual test : Open Turns associates the empirical quantiles of each data from the both numerical samples,

\item the Smirnov test : it tests if both samples (continuous ones only) follow the same distribution. If $F_{n_1}^{*}$ and $F_{n_2}^{*}$ are the empirical cumulative density functions of both samples of size $n_1$ and $n_2$, the Smirnov test evaluates the decision variable :
  \begin{align*}
    D^2 = \displaystyle \sqrt{\frac{n_1n_2}{n_1+n_2}} \sup_{x}|F_{n_1}^{*}(x) - F_{n_2}^{*}(x)|
  \end{align*}
  which tends towards the Kolmogorov distribution. The hypothesis of same distribution is rejected if $D^2$ is too high (depending on the p-value threshold).
\end{itemize}



Details on the QQ-polt and Kolmogorov-Smirnov  tests may be found in the Reference Guide (\extref{ReferenceGuide}{see files Reference Guide - Step B -- Using QQ-plot to compare two samples and Step B -- Comparison of two samples using Kolmogorov-Smirnov test}{stepB}).\\


\requirements{
  \begin{description}
  \item[$\bullet$] two scalar numerical continuous samples : {\itshape sample1, sample2}
  \item[type:]  NumericalSample
  \end{description}
}
             {
               \begin{description}
               \item[$\bullet$] test result : {\itshape resultSmirnov}
               \item[type:] TestResult
               \end{description}
             }

             \textspace\\
             Python script for this UseCase  :

             \inputscript{script_docUC_InputWithData_estSameDist}
