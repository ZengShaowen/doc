% Copyright (C) 2005-2015 Airbus - EDF - IMACS - Phimeca
% Permission is granted to copy, distribute and/or modify this document
% under the terms of the GNU Free Documentation License, Version 1.2
% or any later version published by the Free Software Foundation;
% with no Invariant Sections, no Front-Cover Texts, and no Back-Cover
% Texts.  A copy of the license is included in the section entitled "GNU
% Free Documentation License".
\renewcommand{\etapemethodo}{B}
\renewcommand{\nomfichier}{docref_B21_BayesianCalibration}
\renewcommand{\titrefiche}{Bayesian Calibration}

\Header

\MathematicalDescription{

  \underline{\textbf{Goal}} \vspace{2mm}

  We consider a computer model $h$ ({\em i.e.} a deterministic function) to calibrate:
  \begin{align*}
    \vect{z} = h(\vect{x}, \vect{\theta}_h),
  \end{align*}
  where
  \begin{itemize}
  \item $\vect{x} \in \Rset^{d_x}$ is the input vector;
  \item $\vect{z} \in \Rset^{d_z}$ is the output vector;
  \item $\vect{\theta}_h \in \Rset^{d_h}$ are the unknown parameters of $h$ to calibrate.
  \end{itemize}
  Our goal here is to estimate $\vect{\theta}_h$, based on a certain set of $n$ inputs $(\vect{x}^1, \ldots, \vect{x}^n)$
  (an experimental design) and some associated observations $(\vect{y}^1, \ldots, \vect{y}^n)$ which are regarded
  as the realizations of some random vectors $(\vect{Y}^1, \ldots, \vect{Y}^n)$, such that, for all $i$,
  the distribution of $\vect{Y}^i$ depends on $\vect{z}^i = h(\vect{x}^i, \vect{\theta}_h)$.
  Typically, $\vect{Y}^i = \vect{z}^i + \vect{\varepsilon}^i$ where $\vect{\varepsilon}^i$ is a random measurement error.

  For the sake of clarity, lower case letters are used for both random variables and realizations in the following (the notation does not
  distinguish the two anymore), as usual in the bayesian literature.

  In fact, the bayesian procedure which is implemented allows to infer some unknown parameters $\vect{\theta}\in\Rset^{d_\theta}$
  from some data $\mat{y} = (\vect{y}^1, \ldots, \vect{y}^n)$ as soon as the conditional distribution
  of each $\vect{y}^i$ given $\vect{\theta}$ is specified. Therefore $\vect{\theta}$ can be made up
  with some computer model parameters $\vect{\theta}_h$ together with some others $\vect{\theta}_\varepsilon$:
  $\vect{\theta} = \Tr{(\Tr{\vect{\theta}_h}, \Tr{\vect{\theta}_\varepsilon})}$. For example, $\vect{\theta}_\varepsilon$ may
  represent the unknown standard deviation $\sigma$ of
  an additive centered gaussian measurement error affecting the data (see the example hereafter).
  Besides the procedure can be used to estimate the parameters of a distribution from direct observations (no computer
  model to calibrate: $\vect{\theta} = \vect{\theta}_\varepsilon$).

  More formally, the likelihood $L(\mat{y} | \vect{\theta})$
  is defined by, firstly, a family $\{\cP_{\vect{w}}, \vect{w} \in \Rset^{d_w}\}$ of probability distributions
  parametrized by $\vect{w}$, which is specified in practice by a conditional distribution $f(.|\vect{w})$ given $\vect{w}$ ($f$ is a PDF
  or a probability mass function), and, secondly, a function $g:\Rset^{d_\theta} \longrightarrow \Rset^{n\,d_w}$ such that
  $g(\theta) = \Tr{(\Tr{g^1(\vect{\theta})}, \ldots, \Tr{g^n(\vect{\theta})})}$ which enables to express
  the parameter $\vect{w}^i$ of the ith observation $\vect{y}^i \sim f(.|\vect{w}^i)$ in function of
  $\vect{\theta}$: $g^i(\vect{\theta}) = \vect{w}^i$ thus $\vect{y}^i \sim f(.|g^i(\vect{\theta}))$ and
  \begin{align*}
    L(\mat{y} | \vect{\theta}) = \prod_{i=1}^n f(\vect{y}^i|g^i(\vect{\theta})).
  \end{align*}

  Considering the issue of the calibration of some computer model parameters $\vect{\theta}_h$,
  the full statistical model can be seen as a two-level hierarchical model, with a single level of latent variables $\vect{z}$.
  A classical example is given by the nonlinear Gaussian regression model:
  \begin{align*}
    y_i &=& h(\vect{x}_i|\vect{\theta}_h) + \varepsilon_i,
    \mbox{ where } \varepsilon_i \stackrel{i.i.d.}{\sim} \cN(0, \sigma^2),
    \quad i = 1,\ldots, n.
  \end{align*}
  It can be implemented with $f(.|\Tr{(\mu, \sigma)})$ the PDF of the gaussian distribution $\cN(\mu, \sigma^2)$,
  with $g^i(\vect{\theta}) = \Tr{(h(\vect{x}^i, \vect{\theta}_h), \:\sigma)}$, and with
  $\vect{\theta} = \vect{\theta}_h$, respectively $\vect{\theta} = \Tr{(\Tr{\vect{\theta}_h}, \sigma)}$,
  if $\sigma$ is considered known, respectively unknown.

  Given a distribution modelling the uncertainty on $\vect{\theta}$ prior to the data,
  Bayesian inference is used to perform the inference of $\vect{\theta}$, hence the name Bayesian calibration.

  \vspace{2mm}

  \underline{\textbf{Principle}} \vspace{2mm}

  Contrary to the maximum likelihood approach described in \otref{docref_B21_MaximumLikelihood}{Maximum Likelihood Principle},
  which provides a single `best estimate' value $\hat{\vect{\theta}}$, together with confidence bounds accounting for the uncertainty
  remaining on the true value $\vect{\theta}$, the Bayesian approach derives a full distribution of possible values for $\vect{\theta}$
  given the available data $\mat{y}$. Known as the {\em posterior distribution} of $\vect{\theta}$ given the data $\mat{y}$,
  its density can be expressed according
  to Bayes' theorem:
  \begin{align}\label{eq:post_distribution}
    \pi(\vect{\theta} | \mat{y}) &=& \frac{L(\mat{y} | \vect{\theta}) \times \pi(\vect{\theta})}{m(\mat{y})},
  \end{align}
  where
  \begin{itemize}
  \item $L(\mat{y} | \vect{\theta})$ is the (data) likelihood;
  \item $\pi(\vect{\theta})$ is the so-called {\em prior distribution} of $\vect{\theta}$ (with support $\Theta$),
    which encodes all possible $\vect{\theta}$ values weighted by their prior probabilities, before consideration of any experimental data
    (this allows for instance to incorporate expert information or known physical constraints on the calibration parameter)
  \item $m(\mat{y})$ is the marginal likelihood:
    \begin{align*}
      m(\mat{y}) &=& \displaystyle\int_{\vect{\theta}\in\Theta} L(\mat{y} | \vect{\theta}) \pi(\vect{\theta}) d\vect{\theta},
    \end{align*}
    which is the necessary normalizing constant ensuring that the posterior density integrates to $1$.
  \end{itemize}

  Except in very simple cases, (\ref{eq:post_distribution}) has, in general, no closed form. Thus, it must be approximated, either using numerical integration when the parameter space dimension $d_\theta$ is low,
  or more generally through stochastic sampling techniques known as Monte-Carlo Markov-Chain (MCMC) methods.
  See \otref{docref_B21_MetropolisHastings}{The Metropolis-Hastings Algorithm}.

  The following bibliographical references provide main starting points for further study of  this method:
  \begin{itemize}
  \item Berger, J.O. (1985). "Statistical Decision Theory and Bayesian Analysis", Springer.
  \item Marin J.M. \& Robert C.P. (2007) "Bayesian Core: A Practical Approach to Computational Bayesian Statistics", Springer.
  \end{itemize}
}
{
-
}
