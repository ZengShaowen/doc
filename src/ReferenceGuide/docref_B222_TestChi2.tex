% Copyright (C) 2005-2015 Airbus - EDF - IMACS - Phimeca
% Permission is granted to copy, distribute and/or modify this document
% under the terms of the GNU Free Documentation License, Version 1.2
% or any later version published by the Free Software Foundation;
% with no Invariant Sections, no Front-Cover Texts, and no Back-Cover
% Texts.  A copy of the license is included in the section entitled "GNU
% Free Documentation License".
\renewcommand{\etapemethodo}{B}
\renewcommand{\nomfichier}{docref_B222_TestChi2}
\renewcommand{\titrefiche}{Chi-squared goodness of fit test}

\Header

\MathematicalDescription{

  \underline{\textbf{Goal}} \vspace{2mm}

  This method deals with the modelling of a probability distribution of a random vector $\vect{X} = \left( X^1,\ldots,X^{n_X} \right)$. It seeks to verify the compatibility between a sample of data $\left\{ \vect{x}_1,\vect{x}_2,\ldots,\vect{x}_N \right\}$ and a candidate probability distribution previous chosen. OpenTURNS enables the use of the $\chi^2$  Goodness-of-Fit test to answer this question in the one dimensional case $n_X =1$, and with a discrete distribution.
  \vspace{2mm}

  \underline{\textbf{Principle}} \vspace{2mm}

  Let us limit the case to $n_X = 1$. Thus we denote $\vect{X} = X^1 = X$. We also note that as we are considering discrete distributions i.e. those for which the possible values of $X$ belong to a discrete set $\cE$, the candidate distribution is characterised by the probabilities $\left\{ p(x;\vect{\theta}) \right\}_{x \in \cE}$.

  The chi squared test is based on the fact that if the candidate distribution is appropriate, the number of values in the sample {x1, x2, ..., xN} that are equal to $x$ should be on average equal to $N p(x;\vect{\theta})$. The idea is therefore to compare the "theoretical values" with the actual observed values. This comparison is performed with the aid of the following "distance".
  \begin{align*}
    \widehat{D}_N^2 = \sum_{x \in \cE_N} \frac{\left(Np(x)-n(x)\right)^2}{n(x)}
  \end{align*}
  where $\cE_N$ denotes the elements of $\cE$ which have been observed at least once in the data sample and where $n(x)$ denotes the number of data values in the sample that are equal to $x$.\\

  The probability distribution of the distance $\widehat{D}_N^2$ is asymptotically known (i.e. as the size of the sample tends to infinity), and this asymptotic distribution does not depend on the candidate distribution being tested. If $N$ is sufficiently large, this means that for a probability $\alpha$, one can calculate the threshold / critical value) $d_\alpha$ such that:
  \begin{itemize}
  \item if  $\widehat{D}_N>d_{\alpha}$, we reject the candidate distribution with a risk of error $\alpha$,
  \item if  $\widehat{D}_N \leq d_{\alpha}$, the candidate distribution is considered acceptable.
  \end{itemize}

  An important notion is the so-called "$p$-value" of the test. This quantity is equal to the limit error probability $\alpha_\textrm{lim}$ under which the candidate distribution is rejected. Thus, the candidate distribution will be accepted if and only if $\alpha_\textrm{lim}$ is greater than the value $\alpha$ desired by the user. Note that the higher $\alpha_\textrm{lim} - \alpha$, the more robust the decision.
}
{
}


\Methodology{
  This method is used in step B "Quantifying Sources of Uncertainty", to verify if the probability distribution is appropriate to describe the uncertainty of a component $X^i$ of the vector of unknown variables defined in step A "Specifying Criteria and the Case Study".\\

  \textbf{Input data:}\\
  $\left\{x_1,\ldots,x_N\right\}$ : data sample\\
  \textit{Distribution}: probability distribution that we are testing for goodness-of-fit \\

  \textbf{Parameters:}\\
  $\alpha$ : Level of significance for the test \\

  \textbf{Outputs:}\\
  \textit{Result}: Binary variable specifying whether the candidate distribution is rejected (0) or not (1) \\
  $\alpha_\textrm{lim}$ : $p$-value of the test

}
{
  The test is suitable for discrete distributions. It cannot be used for continuous distributions except by means of an arbitrary discretisation of possible values of $X$, an important source of potential error. Readers interested in Goodness of Fit tests for continuous variables are referred to \otref{docref_B222_TestKS}{Kolmogorov-Smirnov test}, \otref{docref_B222_TestCVM}{Cramer-Von Mises test}, \otref{docref_B222_TestAD}{Anderson-Darling test} in the reference documentation.\\

  Even for discrete distributions, certain precautions must be taken when using this test. Firstly, the critical value $d_\alpha$ is only valid for a sufficiently large sample size. No rule exists to determine the minimum number of data values necessary in order to use this test; it is often thought, however, that the approximation is reasonable when $N$ is of the order of a few dozen. But whatever the value of $N$, the distance -- and similarly the $p$-value -- remains a useful tool for comparing different probability distributions to a sample. The distribution which minimizes $\widehat{D}_N$ -- or maximizes the $p$-value -- will be of interest to the analyst.

  On the other hand, the calculation of $d_\alpha$ and of the $p$-value should in theory be modified if we are testing the goodness of fit of a parametric model and if the parameters of the candidate distribution have been estimated from the same sample. The current version of OpenTURNS, however, does not permit such a modification, and so the results must be used with care when the $p$-value $\alpha_\textrm{lim}$ and the desired error risk $\alpha$ are very close.\\

  The following bibliographical references provide main starting points for further study of this method:
  \begin{itemize}
  \item Saporta, G. (1990). "Probabilités, Analyse de données et Statistique", Technip
  \item Dixon, W.J. \& Massey, F.J. (1983) "Introduction to statistical analysis (4th ed.)", McGraw-Hill
    % \item NIST/SEMATECH e-Handbook of Statistical Methods, http://www.itl.nist.gov/div898/handbook/
  \item D'Agostino, R.B. and Stephens, M.A. (1986). "Goodness-of-Fit Techniques", Marcel Dekker, Inc., New York.
  \item Bhattacharyya, G.K., and R.A. Johnson, (1997). "Statistical Concepts and Methods", John Wiley and Sons, New York.
  \item Sprent, P., and Smeeton, N.C. (2001). "Applied Nonparametric Statistical Methods -- Third edition", Chapman \& Hall
    % \item Burnham, K.P., and Anderson, D.R (2002). "Model Selection and Multimodel Inference: A Practical Information Theoretic Approach", Springer
\end{itemize}}
