% Copyright 2005-2016 Airbus-EDF-IMACS-Phimeca
% Permission is granted to copy, distribute and/or modify this document
% under the terms of the GNU Free Documentation License, Version 1.2
% or any later version published by the Free Software Foundation;
% with no Invariant Sections, no Front-Cover Texts, and no Back-Cover
% Texts.  A copy of the license is included in the section entitled "GNU
% Free Documentation License".
\renewcommand{\etapemethodo}{C}
\renewcommand{\nomfichier}{docref_C322_QuasiMonteCarlo}
\renewcommand{\titrefiche}{Quasi Monte Carlo}

\Header

\MathematicalDescription{

  \underline{\textbf{Goal}}\\
  Let us note $\cD_f = \{\ux \in \Rset^{n} \: | \:  g(\ux,\underline{d}) \leq 0\}$.
  The goal is to estimate the following probability:
  \begin{eqnarray*}\label{integ}
    P_f &=& \int_{\cD_f} f_{\uX}(\ux)d\ux\\
    &=& \int_{\Rset^{n}} \mathbf{1}_{\{g(\ux,\underline{d}) \leq 0 \}}f_{\uX}(\ux)d\ux\\
    &=& \Prob {\{\:g(\uX,\underline{d}) \leq 0 \}}
  \end{eqnarray*}
  Quasi-Monte Carlo is a technique which approximates the integral (\ref{integ})   using low discrepancy sequences  $\{\vect{x}^1, ..., \vect{x}^N\}$ instead of randomly generated sequences, as follows :
  \begin{align*}
    P_f \approx \frac{1}{N}\,\sum_{i=1}^N  \mathbf{1}_{\cD_f}(\ux_i) f(\ux_i).
  \end{align*}
  To have information on low discrepancy sequences, refer to \otref{docref_C_LowDiscrepancySequence}{Low Discrepancy Sequence} -- see page \pageref{docref_C_LowDiscrepancySequence}. \\


  \underline{\textbf{Principles}}\\

  In general, the integral of a function $f$  on $\Delta = [0,1]^s$ can be approximated by using some low discrepancy sequence $\{\vect{x}_1, \hdots, \vect{x}_N\}$  as follows :
  \begin{align*}
    \int_{\Delta} f(\vect{u})\,d\vect{u} \approx \frac{1}{N}\,\sum_{i=1}^N f(\vect{x}_i).
  \end{align*}

  The low discrepancy sequence is generated on $\Delta$ according to the Lebesgue measure then may be transformed to any measure $\mu$ thanks to the inverse CDF technique in order to approximate the integral :
  \begin{align*}
    \int_{\Rset^s} f(\vect{u})\,d\vect{u} \approx \frac{1}{N}\,\sum_{i=1}^N f(\vect{x}_i).
  \end{align*}


  As the sequence is not randomly generated, we can not use the Central Limit Theorem in order to control the quality of the approximation. This quality is given by the  Koksma-Hlawka inequality that we recall here :
  \begin{align*}
    \left\lvert \frac{1}{N}\sum_{i=1}^Nf(\vect{x}_i) - \int_If(\vect{x})d\vect{x} \right\rvert \le Var(f)D^N(\vect{x}_1, ..., \vect{x}_N)
  \end{align*}

  where $D^N(\vect{x}_1, ..., \vect{x}_N)$ is the discrepancy of the sequence $\{\vect{x}_1, \hdots, \vect{x}_N\}$.\\

  For Halton, Inverse Halton and Sobol sequences, we have :
  \begin{align*}
    D^N = O\biggl(\frac{\log^s{N}}{N}\biggr)
  \end{align*}
  Thus, asymptotically the error converges in $O\biggl(\frac{\log^s{N}}{N}\biggr)$, instead of $O\biggl(\frac{1}{\sqrt{N}}\biggr)$ for traditional Monte Carlo. The convergence rate depends on the dimension $s$ so one must have $N >> e^s$.
}
{
  % other notations
  --
}

\Methodology{
  This method is part of the step C of the global methodology. It requires the specification of the joined probability density function of the input variables and the definition of the threshold. The PDF must have an independent copula.
}
            {This method \textit{a priori} provides an asymptotically higher convergence rate than traditional Monte Carlobut no general rule can guarantee a better efficiency of the Quasi-Monte Carlo sampling than the classical Monte Carlo sampling.\\
              The advantages of Quasi-Monte Carlo sampling tend to disappear with the increase of the number of dimensions of the input variables. \\
              It is recommended to use the Quasi-Monte Carlo technique with high sampling sizes or with very low dimensionality.\\
              This method is valid only if the input random variables are independent.\\
              A fruitful link towards the global approach can be established with the files :
              \begin{itemize}
              \item \otref{docref_C321_MonteCarloStd}{Monte Carlo method to evaluate a probability to exceed a threshold},
              \item   \otref{docref_C322_LHS}{Latin Hypercube Sampling to evaluate a probability to exceed a threshold},
              \item  \otref{docref_C322_TI}{Importance Sampling to evaluate a probability to exceed a threshold} coming from the methodology.
              \end{itemize}
            }
