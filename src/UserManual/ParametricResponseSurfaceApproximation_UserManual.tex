% Copyright (C) 2005-2015 Airbus - EDF - IMACS - Phimeca
% Permission is granted to copy, distribute and/or modify this document
% under the terms of the GNU Free Documentation License, Version 1.2
% or any later version published by the Free Software Foundation;
% with no Invariant Sections, no Front-Cover Texts, and no Back-Cover
% Texts.  A copy of the license is included in the section entitled "GNU
% Free Documentation License".


\newpage
\extanchor{paraRespSurface}
\section{Response Surface : Parametric Approximation}

%\index{Response Surface : Parametric Approximation!Taylor approximation}
\subsection{Taylor approximation}

%\index{Response Surface : Parametric Approximation!Taylor approximation!LinearTaylor}
\index{LinearTaylor}
\subsubsection{LinearTaylor}

\begin{description}

\item[Usage:] \textit{LinearTaylor(center, function)}
\bigskip

\item[Arguments:]  \rule{0pt}{1em}
\begin{description}
\item \textit{center}: a NumericalPoint, the point where the Taylor expansion of the function \textit{function} is performed
\item \textit{function}: a NumericalMathFunction, the function to be approximated.
\end{description}

\item[Value:] a LinearTaylor

\item[Some methods :]  \rule{0pt}{1em}
\begin{description}

\item \textit{getInputFunction}
\begin{description}
\item[Usage:] \textit{getInputFunction}
\item[Arguments:] none
\item[Value:] a NumericalMathFunction, the function \textit{function}
\end{description}
\bigskip

\item \textit{getName}
\begin{description}
\item[Usage:] \textit{getName()}
\item[Arguments:] none
\item[Value:] a string, the name of the LinearTaylor
\end{description}
\bigskip

\item \textit{getCenter}
\begin{description}
\item[Usage:] \textit{getCenter()}
\item[Arguments:] none
\item[Value:] a NumericalPoint, around which the approximation has been made : \textit{center}
\end{description}
\bigskip

\item \textit{run}
\begin{description}
\item[Usage:] \textit{run()}
\item[Arguments:] none
\item[Value:] it performs the linear Taylor expansion around \textit{center}
(while this method has not been executed, only
\textit{getInputFunction}, \textit{getName} and \textit{setName} methods can be used)
\end{description}
\bigskip

\item \textit{getConstant}
\begin{description}
\item[Usage:] \textit{getConstant()}
\item[Arguments:] none
\item[Value:] a NumericalPoint, the constant vector of the approximation, equal to \textit{function(center)}
\end{description}
\bigskip

\item \textit{getLinear}
\begin{description}
\item[Usage:] \textit{getLinear()}
\item[Arguments:] none
\item[Value:] a Matrix, the gradient of the function \textit{function} at the point \textit{center} (the transposition of the jacobian matrix)
\end{description}
\bigskip


\item \textit{getResponseSurface}
\begin{description}
\item[Usage:] \textit{getResponseSurface()} %
\item[Arguments:] none
\item[Value:] a NumericalMathFunction, an approximation of the function \textit{function} by a linear Taylor expansion at the \textit{center}
\end{description}
\bigskip

\end{description}

\item[Links:] \rule{0pt}{1em}
\extref{ReferenceGuide}{see ReferenceGuide - Linear and Quadratic Taylor Expansions}{docref_SurfRep_Taylor}
\end{description}

The methods \textit{getInputFunction}, \textit{getName}, \textit{getCenter} have their associated \textit{setMethod}.

% -=============================================================


\newpage
%\index{Response Surface : Parametric Approximation!Taylor approximation!QuadraticTaylor}
\index{QuadraticTaylor}
\subsubsection{QuadraticTaylor}

\begin{description}

\item[Usage:] \textit{QuadraticTaylor(center, function)}

\item[Arguments:]  \rule{0pt}{1em}
\begin{description}
\item \textit{center}: a NumericalPoint, the point where the quadratic Taylor expansion of
the function \textit{function} is performed
\item \textit{function}: a NumericalMathFunction, the function to be approximated : the gradient and  hessian of the NumericalMathFunction must be defined.
\end{description}

\item[Value:] a QuadraticTaylor
\bigskip

\item[Some methods :]  \rule{0pt}{1em}
\begin{description}

\item \textit{getInputFunction}
\begin{description}
\item[Usage:] \textit{getInputFunction}
\item[Arguments:] none
\item[Value:] a NumericalMathFunction, the function \textit{function}
\end{description}
\bigskip

\item \textit{getName}
\begin{description}
\item[Usage:] \textit{getName()}
\item[Arguments:] none
\item[Value:] a string, the name of the QuadraticTaylor
\end{description}
\bigskip

\item \textit{getCenter}
\begin{description}
\item[Usage:] \textit{getCenter()}
\item[Arguments:] none
\item[Value:] a NumericalPoint, around which the approximation has been made : \textit{center}
\end{description}
\bigskip

\item \textit{run}
\begin{description}
\item[Usage:] \textit{run()}
\item[Arguments:] none
\item[Value:] it performs the Quadratic Taylor expansion around \textit{center}
(while this method has not been executed, only
\textit{getInputFunction}, \textit{getName} and \textit{setName} methods can be used)
\end{description}
\bigskip

\item \textit{getConstant}
\begin{description}
\item[Usage:] \textit{getConstant()}
\item[Arguments:] none
\item[Value:] a NumericalPoint, the constant vector of the approximation, equal to \textit{function(center)}
\end{description}
\bigskip

\item \textit{getLinear}
\begin{description}
\item[Usage:] \textit{getLinear()}
\item[Arguments:] none
\item[Value:] a Matrix, the gradient of the function \textit{function} at the point \textit{center} (the transposition of the jacobian matrix)
\end{description}
\bigskip

\item \textit{getQuadratic}
\begin{description}
\item[Usage:] \textit{getQuadratic()}
\item[Arguments:] none
\item[Value:] a SymmetricTensor which contains the 0.5 *  transposition of the hessian values of \textit{function} at \textit{center}
\end{description}
\bigskip

\item \textit{getResponseSurface}
\begin{description}
\item[Usage:] \textit{getResponseSurface()}
\item[Arguments:] none
\item[Value:] a NumericalMathFunction, an approximation of the function \textit{function} by a Quadratic Taylor expansion at  \textit{center}
\end{description}
\end{description}

\item[Links:] \rule{0pt}{1em}
\extref{ReferenceGuide}{see ReferenceGuide - Linear and Quadratic Taylor Expansions}{docref_SurfRep_Taylor}

The methods \textit{getInputFunction}, \textit{getName}, \textit{getCenter} have their associated \textit{setMethod}.

\end{description}



% ===================================================================


\newpage
%\index{Response Surface : Parametric Approximation!Least squares approximation}
\subsection{Least squares approximation}

%\index{Response Surface : Parametric Approximation!Least squares approximation!LinearLeastSquares}
\index{LinearLeastSquares}

\subsubsection{LinearLeastSquares}

\begin{description}

\item[Usage:] \rule{0pt}{1em}
\begin{description}
\item \textit{LinearLeastSquares(dataIn, function)}
\item \textit{LinearLeastSquares(dataIn,dataOut)}
\end{description}

\item[Arguments:]  \rule{0pt}{1em}
\begin{description}
\item \textit{dataIn}: a NumericalSample, the input variables
\item \textit{function}: a NumericalMathFunction, the function to be approximated
\item \textit{dataOut}: a NumericalSample, the output variables
\end{description}

\item[Value:] a LinearLeastSquares, the linear least squares approximation between :
\begin{description}
\item the two samples  \textit{dataIn} and \textit{dataOut} in the case of the second usage
\item the two samples  \textit{dataIn} and \textit{function(dataIn)} in the case of the first usage
\end{description}

\item[Some methods :]  \rule{0pt}{1em}
\begin{description}

\item \textit{getInputFunction}
\begin{description}
\item[Usage:] \textit{getInputFunction()}
\item[Arguments:] none
\item[Value:] a NumericalMathfunctiontion the \textit{function} parameter in the case of the first usage
\end{description}
\bigskip

\item \textit{getDataIn}
\begin{description}
\item[Usage:] \textit{getDataIn()}
\item[Arguments:] none
\item[Value:] a NumericalSample, the \textit{dataIn} parameter
\end{description}
\bigskip

\item \textit{getName}
\begin{description}
\item[Usage:] \textit{getName()}
\item[Arguments:] none
\item[Value:] a string, the name of the LinearLeastSquares
\end{description}
\bigskip


\item \textit{run}
\begin{description}
\item[Usage:] \textit{run()}
\item[Arguments:] none
\item[Value:] it performs the linear least squares approximation (while this method has not been executed, only \textit{getInputfunctiontion}, \textit{getDataIn}, \textit{getName} and \textit{setName} methods can be used)
\end{description}
\bigskip


\item \textit{getDataOut}
\begin{description}
\item[Usage:] \textit{getDataIn()}
\item[Arguments:] none
\item[Value:] a NumericalSample, it returns the ouput variable :
\begin{description}
\item in the case of the first usage, it corresponds to the values of the function \textit{function} at the input variables \textit{dataIn}: \textit{function(dataIn)}
\item in the case of the second usage, it corresponds to  \textit{dataOut}
\end{description}
\end{description}
\bigskip

\item \textit{getLinear}
\begin{description}
\item[Usage:] \textit{getLinear()} %
\item[Arguments:] none
\item[Value:] a Matrix, the gradient of the function \textit{function} at the point \textit{center} (the transposition of the jacobian matrix)
\end{description}
\bigskip

\item \textit{getResponseSurface}
\begin{description}
\item[Usage:] \textit{getResponseSurface()} %
\item[Arguments:] none
\item[Value:] a NumericalMathFunction, an approximation of the function \textit{function} by Linear Least Squares
\end{description}

\end{description}

\item[Links:]
\extref{ReferenceGuide}{see ReferenceGuide - Least square}{docref_SurfRep_IntegLeastSquare}
\end{description}

The methods \textit{getInputFunction}, \textit{getName}, \textit{getDataIn} have their associated \textit{setMethod}.

% ===================================================================


\newpage
%\index{Response Surface : Parametric Approximation!Least squares approximation!QuadraticLeastSquares}
\index{QuadraticLeastSquares}
\subsubsection{QuadraticLeastSquares}

\begin{description}

\item[Usage:]
\begin{description}
\item \textit{QuadraticLeastSquares(dataIn, function)}
\item \textit{QuadraticLeastSquares(dataIn, dataOut)}
\end{description}

\item[Arguments:]  \rule{0pt}{1em}
\begin{description}
\item \textit{dataIn}: a NumericalSample, the input variables
\item \textit{function}: a NumericalMathFunction, the function to be approximated
\item \textit{dataOut}: a NumericalSample, the output variables
\end{description}

\item[Value:] a QuadraticLeastSquares, the quadratic least squares approximation between :
\begin{description}
\item the two samples  \textit{dataIn} and \textit{dataOut} in the case of the second usage
\item the two samples  \textit{dataIn} and \textit{function(dataIn)} in the case of the first usage
\end{description}

\item[Some methods :]  \rule{0pt}{1em}
\begin{description}

\item \textit{getInputFunction}
\begin{description}
\item[Usage:] \textit{getInputFunction}
\item[Arguments:] none
\item[Value:] a NumericalMathFunction, the function \textit{function}
\end{description}
\bigskip

\item \textit{getName}
\begin{description}
\item[Usage:] \textit{getName()}
\item[Arguments:] none
\item[Value:] a string, the name of the QuadraticLeastSquares
\end{description}
\bigskip

\item \textit{run}
\begin{description}
\item[Usage:] \textit{run()}
\item[Arguments:] none
\item[Value:] it performs the quadratic least squares approximation
(while this method has not been executed, only
\textit{getInputFunction}, \textit{getName} and \textit{setName} methods can be used)
\end{description}
\bigskip

\item \textit{getDataOut}
\begin{description}
\item[Usage:] \textit{getDataIn()}
\item[Arguments:] none
\item[Value:] a NumericalSample, it returns the ouput variable :
\begin{description}
\item in the case of the first usage, it corresponds to the values of the function \textit{function} at the input variables \textit{dataIn}: \textit{function(dataIn)}
\item in the case of the second usage,  it corresponds to  \textit{dataOut}
\end{description}
\end{description}
\bigskip

\item \textit{getConstant}
\begin{description}
\item[Usage:] \textit{getConstant()}
\item[Arguments:] none
\item[Value:] a NumericalPoint, the constant vector of the approximation, equal to \textit{function(center)}
\end{description}
\bigskip

\item \textit{getLinear}
\begin{description}
\item[Usage:] \textit{getLinear()}
\item[Arguments:] none
\item[Value:] a Matrix, the linear matrix of the approximation
\end{description}
\bigskip


\item \textit{getQuadratic}
\begin{description}
\item[Usage:] \textit{getQuadratic()}
\item[Arguments:] none
\item[Value:] a SymmetricTensor, the quadratic term of the approximation
\end{description}

\item \textit{getResponseSurface}
\begin{description}
\item[Usage:] \textit{getResponseSurface()} %
\item[Arguments:] none
\item[Value:] a NumericalMathFunction, an approximation of the function \textit{function} by Quadratic Least Squares
\end{description}

\end{description}

\item[Links:]\rule{0pt}{1em}
\extref{ReferenceGuide}{see ReferenceGuide - Least square}{docref_SurfRep_IntegLeastSquare}
\end{description}

The methods \textit{getInputFunction}, \textit{getName}, \textit{getDataIn} have their associated \textit{setMethod}.

% ===================================================================
