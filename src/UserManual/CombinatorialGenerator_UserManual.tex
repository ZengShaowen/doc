% Copyright (C) 2005-2015 Airbus - EDF - IMACS - Phimeca
% Permission is granted to copy, distribute and/or modify this document
% under the terms of the GNU Free Documentation License, Version 1.2
% or any later version published by the Free Software Foundation;
% with no Invariant Sections, no Front-Cover Texts, and no Back-Cover
% Texts.  A copy of the license is included in the section entitled "GNU
% Free Documentation License".

\extanchor{combinatiorialGen}
\newpage \section{Combinatorial Generators}

\index{CombinatorialGenerator}
\subsection{CombinatorialGenerator}

\begin{description}

\item[Usage:] \textit{CombinatorialGenerator(combGenImplentation)}

\item[Arguments:]  \textit{combGenImplentation}: a CombinatorialGeneratorImplementation.

\item[Value:] a CombinatorialGenerator

\item[Some methods:]  \rule{0pt}{1em}
\begin{description}
\item \textit{generate}
\begin{description}
\item[Usage:] \textit{generate()}
\item[Value:] an IndicesCollection, the collection of all the possible values of the combinatorial generator as a set of nonnegative integer values stored into an Indices.
\end{description}
\end{description}

\item[Links]  \rule{0pt}{1em}
\extref{ReferenceGuide}{see Reference Guide - C11 MinMaxPlanExp}{docref_C11_MinMax}

\end{description}

% =================================================

\newpage
\index{Combinations}
\subsubsection{Combinations}

This class inherits from the CombinatorialGenerator class.\\

\begin{description}

\item[Usage:] \rule{0pt}{1em}
\begin{description}
\item \textit{Combinations()}
\item \textit{Combinations(k, n)}
\end{description}

\item[Arguments:]  \rule{0pt}{1em}
\begin{description}
\item $k$ : an UnsignedLong, the cardinal of the subsets
\item $n$ : an UnsignedLong, the cardinal of the base set
\end{description}

\item[Value:] a Combinations generator
\begin{description}
\item In the first usage, the generator is built using the default values $k=1$, $n=1$.
\item In the second usage, the generator produces all the subsets with $k$ elements of a base set with $n$ elements. The subsets are produced as a collection of Indices in lexical order, the elements of each subset being sorted in increasing order.
\end{description}

\item[Details:]  \rule{0pt}{1em}
\begin{description}
\item Number of indices generated : $\dfrac{n!}{k!(n-k)!}$
\item The combinations generator generates an IndicesCollection where :
\begin{description}
\item the Indices are sorted in lexical order,
\item the components are sorted within a given Indices.

For $(k=2,n=5)$ we get $[[0,1],[0,2],[0,3],[0,4],[1,2],[1,3],[1,4],[2,3],[2,4],[3,4]]$.
\end{description}
\end{description}

\item[Links] \rule{0pt}{1em}
\extref{ReferenceGuide}{see Reference Guide - C11 MinMaxPlanExp}{docref_C11_MinMax}
\end{description}


% =================================================
\newpage
\index{KPermutations}
\subsubsection{KPermutations}

This class inherits from the CombinatorialGenerator class.\\

\begin{description}

\item[Usage:] \rule{0pt}{1em}
\begin{description}
\item \textit{KPermutations()}
\item \textit{KPermutations(n)}
\item \textit{KPermutations(k, n)}
\end{description}

\item[Arguments:]  \rule{0pt}{1em}
\begin{description}
\item $k$ : an UnsignedLong, the cardinal of the origin set
\item $n$ : an UnsignedLong, the cardinal of the goal set
\end{description}

\item[Value:] a KPermutations generator
\begin{description}
\item In the first usage, the generator is built using the default values $k=1$, $n=1$.
\item In the second usage, the generator is built using the value $k=n$.
\item In the third usage, the generator produces all the injective functions from a set with $k$ elements into a set with $n$ elements. If $k=n$ it means all the permutations of a set with $n$ elements.
\end{description}

\item[Details:]  \rule{0pt}{1em}
\begin{description}
\item Number of indices generated : $\dfrac{n!}{(n-k)!}$
\item The combinations generator generates an IndicesCollection that contains all the $k!$ permutations of all the $\dfrac{n!}{k!(n-k)!}$ subsets with $k$ elements of a set with $n$ elements. The subsets are generated in lexical order, and for each subset all the corresponding injective functions are generated in lexical order.

For $k=2, n=4$ we get $[[0,1],[1,0],[0,2],[2,0],[0,3],[3,0],[1,2],[2,1],[1,3],[3,1],[2,3],[3,2]]$.
\end{description}

\item[Links] \rule{0pt}{1em}
\extref{ReferenceGuide}{see Reference Guide - C11 MinMaxPlanExp}{docref_C11_MinMax}
\end{description}


% =================================================
\newpage
\index{Tuples}
\subsubsection{Tuples}

This class inherits from the CombinatorialGenerator class.\\

\begin{description}

\item[Usage:] \rule{0pt}{1em}
\begin{description}
\item \textit{Tuples()}
\item \textit{Tuples(bounds)}
\end{description}

\item[Arguments:]  \rule{0pt}{1em}
\begin{description}
\item \textit{bounds}: an Indices, the cardinal of all the sets forming the cartesian product.
\end{description}

\item[Value:] a Tuples generator
\begin{description}
\item In the first usage, the generator is built using the default values $bounds=[1]$.
\item In the second usage, the generator produces all the indices $[i_0,\hdots,i_{d-1}]$ with $i_k\in\{0,\hdots,bounds[k]-1\}$ where $d=bounds.getDimension()$.
\end{description}

\item[Details:]  \rule{0pt}{1em}
\begin{description}
\item Number of indices generated : $\displaystyle\prod_{k=0}^{d-1}bounds[k]$
\item The tuples generator generates an IndicesCollection that contains all the elements of the Cartesian product $\displaystyle\prod_{k=0}^{d-1}\{0,\hdots,bounds[k]-1\}$.

For $bounds=[3,4]$ we get $[[0,0],[1,0],[2,0],[0,1],[1,1],[2,1],[0,2],[1,2],[2,2],[0,3],[1,3],[2,3]]$.
\end{description}

\item[Links] \rule{0pt}{1em}
\extref{ReferenceGuide}{see Reference Guide - C11 MinMaxPlanExp}{docref_C11_MinMax}
\end{description}


% =================================================
